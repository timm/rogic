\documentclass{article}

%%  Latex generated from POD in document /cygdrive/i/tim/src/pl/reason/mypod_tmp.pl
%%  Using the perl module Pod::LaTeX
%%  Converted on Sat Aug 30 22:29:55 2003


\usepackage{makeidx}
\makeindex


    \usepackage{times}
    \setcounter{tocdepth}{2}
    \title{template.pl}
\usepackage{url}
\author{
Tim Menzies\\
Lane Deptartment of Computer Science and Electrical Engineering\\
West Virginia University\\
Morgantown, WV 26506\\
{\small \url{tim@menzies.us}}}
    \date{}
    \begin{document}
     \maketitle
    \small


\tableofcontents

\clearpage
\section{Template\label{Template}\index{Template}}


Standard Prolog file structure */



/*

\section{Header */\label{Header_}\index{Header */}}


/*

\subsection{Loads\label{Loads}\index{Loads}}


Load some code using "ensure\_loaded". e.g.: */

\begin{verbatim}
 :- ensure_loaded([xx]).
\end{verbatim}


/*



(Warning: there is some quirk in the Prolog loader
 when doing \texttt{goal\_expansion}s. If you are using \texttt{ecg}s
 or \texttt{defs}, then DON'T \texttt{ensure\_loaded} those files;
rather, load them the old-fashioned way: */

\begin{verbatim}
 :- [ecg]. */
\end{verbatim}
\subsection{Operators\label{Operators}\index{Operators}}


Change the Prolog parser using \texttt{op/3}x. e.g.: */

\begin{verbatim}
 :- op(700,xfx,and).
\end{verbatim}


/*

\subsection{Flags\label{Flags}\index{Flags}}


Index predicates (with \texttt{index/1})
or make some dynamic (with \texttt{dynamic/1})
or discontiguous (with \texttt{discontiguous/1})
or .... e.g.: */

\begin{verbatim}
 :- index(myemp(1,1,0,1,00)).
 :- discontiguous defs/2.
 :- mulitfile defs/2.
 :- dynamic defs/2.
\end{verbatim}


/*

\subsection{Hooks\label{Hooks}\index{Hooks}}


Change the Prolog parser using \texttt{term\_expansion/2} and
\texttt{goal\_expansion/2}. e.g. */

\begin{verbatim}
 term_expansion(X and Y,Z) :- xpands(X and Y, Z).
\end{verbatim}
\begin{verbatim}
 goal_expansion(L with M and N,Z) :- tupleCompile(L with M and N,Z).
\end{verbatim}


/*

\subsection{Hacks\label{Hacks}\index{Hacks}}


We won't talk about these. I didn't not write these.
You did not see them. */

\begin{verbatim}
 goal_expansion(true(X,X),true).
\end{verbatim}


/*

\section{Body\label{Body}\index{Body}}


Insert your applciation code here.

\subsection{Utils\label{Utils}\index{Utils}}


Finally, add any general utilities last.  e.g. */

\begin{verbatim}
 barph(X) :- format('W> ~w\n',X), fail. /*
\end{verbatim}
\section{Footer */\label{Footer_}\index{Footer */}}


/*

\subsection{Start-ups\label{Start-ups}\index{Start-ups}}


After all the above is loaded, how do we get this stuff started. */

\printindex

\end{document}
