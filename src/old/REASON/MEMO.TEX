\documentclass{article}

%%  Latex generated from POD in document /cygdrive/i/tim/src/pl/reason/mypod_tmp.pl
%%  Using the perl module Pod::LaTeX
%%  Converted on Sat Aug 30 22:29:51 2003


\usepackage{makeidx}
\makeindex


    \usepackage{times}
    \setcounter{tocdepth}{2}
    \title{template.pl}
\usepackage{url}
\author{
Tim Menzies\\
Lane Deptartment of Computer Science and Electrical Engineering\\
West Virginia University\\
Morgantown, WV 26506\\
{\small \url{tim@menzies.us}}}
    \date{}
    \begin{document}
     \maketitle
    \small


\tableofcontents

\clearpage
\section{Memos\label{Memos}\index{Memos}}


Memoing with inconsistency checking.



Each memo is a pair \texttt{Key=Value}.
If the program generates a completely new \texttt{Value} for
\texttt{Key}, then it is stored.



If the program stumbles on another \texttt{Value} for \texttt{Key},
then this code will reject the new \texttt{Value} if contradicts
the older known \texttt{Value}.



At anytime, the program can ask for the current \texttt{value}.



Internally, the \texttt{Key=Value} pairs are stored as
\texttt{assumption}s since if the program backtracks over
memo creation, the \texttt{Value} is thrown away.
That is, all memos are tentative and can be discarded
if that proves useful.

\section{Header\label{Header}\index{Header}}
\subsection{Loads */\label{Loads_}\index{Loads */}}
\begin{verbatim}
 :- [ecg] . /*
\end{verbatim}
\subsection{Flags */\label{Flags_}\index{Flags */}}
\begin{verbatim}
 :- dynamic       assumption/4. 
 :- dynamic       keyValue/3.
 :- discontiguous keyValue/3.
 :- multifile     keyValue/3.  
 :- index(assumption(1,1,1,0)). /*
\end{verbatim}
\section{Body\label{Body}\index{Body}}
\subsection{Assumptions\label{Assumptions}\index{Assumptions}}


Assumptions are stored as \texttt{assumption(Hash,Key,Value,How)} where:

\begin{description}

\item[\texttt{Hash}] \mbox{}

Allows for fast
access to \texttt{assumption}s with complex terms for \texttt{Key}.


\item[\texttt{Key}] \mbox{}

The \textit{name} of the thing assumed.


\item[\texttt{Value}] \mbox{}

The \textit{value} of the thing assumed.


\item[\texttt{How}] \mbox{}

Some comment  on how we got to this \texttt{assumption}.

\end{description}


*/

\begin{verbatim}
 reset :- retractall(assumption(_,_,_,_)). /*
\end{verbatim}
\subsection{\texttt{keyValue}\label{keyValue}\index{keyValue}}


In order to allow \texttt{assumption}s on arbitrary terms,
the \texttt{keyValue(Term,Index,Key,Value)} predicate inputs some \texttt{Term}
and pulls it apart into its \texttt{Key} and \texttt{Value}. Then
it hashes on the \texttt{Key} to find the \texttt{Index}. */

\begin{verbatim}
 keyValue(Term,Hash,Key,Value) :- 
    once(keyValue(Term,Key,Value)), hash_term(Key,Hash). /*
\end{verbatim}


To customize its behavior, add \texttt{keyValue/3} facts: */

\begin{verbatim}
 keyValue(Key=Value,Key,Value).
 keyValue(Term,Term,t). /*
\end{verbatim}
\subsection{Assume\label{Assume}\index{Assume}}


With all that defined, now we can assume things: */

\begin{verbatim}
 assume(X,_)   :- keyValue(X,H,In,Out), assumption(H,In,Old,_),!,Out=Old.
 assume(X,How) :- keyValue(X,H,In,Out), bassert(assumption(H,In,Out,How)). /*
\end{verbatim}
\subsection{Memo\label{Memo}\index{Memo}}


That's all under-the-hood stuff. The main driver of the memo system is
\texttt{memo(Goal,Results)}. */

\begin{verbatim}
 memo(Goal,Memos) :-
    status(Memos,New), 
    (New=0 -> true; Goal, ok(Memos,byRule)).
\end{verbatim}
\begin{verbatim}
 def(status,[contradictions,agreements, new]).
\end{verbatim}
\begin{verbatim}
 status(Memos,New) :- status(Memos,New,_,_).
\end{verbatim}
\begin{verbatim}
 status(L,Flag) --> in status, 
    statusReset, 
    statusRun(L),
    the new=Flag.
\end{verbatim}
\begin{verbatim}
 statusReset--> in status, the contradictions:=0, the agreements:=0, the new:=0.
\end{verbatim}
\begin{verbatim}
 statusRun([]) -->[].
 statusRun([Term|Terms]) -->  in status,
    {keyValue(Term,Hash,Key,Value)},
    statusStep(Hash,Key,Value),
    the contradictions < 1,
    statusRun(Terms). /*
\end{verbatim}


Three cases:

\begin{enumerate}

\item 1

Old \texttt{Key} is missing: we have a new key to find. */

\begin{verbatim}
 statusStep(H,K,_) --> in status, {\+ assumption(H,K,_,_)   },  +the new. /*
\end{verbatim}

\item 1

Old \texttt{Key} is present, new value is bound: check that old=new,  otherwise we 
should note a contradiction. */

\begin{verbatim}
 statusStep(H,K,V) --> in status, {ground(V), assumption(H,K,V,_ )},          +the agreements. 
 statusStep(H,K,V) --> in status, {ground(V), assumption(H,K,V0,_), V0 \= V}, +the contradictions. /*
\end{verbatim}

\item 1

Old \texttt{Key} is present, new value is unbound: bind new value to old value.*/

\begin{verbatim}
 statusStep(H,K,V) --> in status, {var(V), assumption(H,K,V,_)}  , +the agreements. /*
\end{verbatim}
\end{enumerate}
\begin{verbatim}
 */
\end{verbatim}
\begin{verbatim}
 ok([],_).
 ok([H|T],How) :- assume(H,How),ok(T,How). /*
\end{verbatim}
\subsection{Utils\label{Utils}\index{Utils}}


Ye olde backtrackable assert. Good for recording information
about assumptions, then forgetting about them if that don't
pan out.
*/

\begin{verbatim}
 bassert(X) :- assert(X).
 bassert(X) :- retract(X),fail.
\end{verbatim}
\printindex

\end{document}
