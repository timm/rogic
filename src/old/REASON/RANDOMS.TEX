\documentclass{article}

%%  Latex generated from POD in document /cygdrive/i/tim/src/pl/reason/mypod_tmp.pl
%%  Using the perl module Pod::LaTeX
%%  Converted on Sat Aug 30 22:29:52 2003


\usepackage{makeidx}
\makeindex


    \usepackage{times}
    \setcounter{tocdepth}{2}
    \title{template.pl}
\usepackage{url}
\author{
Tim Menzies\\
Lane Deptartment of Computer Science and Electrical Engineering\\
West Virginia University\\
Morgantown, WV 26506\\
{\small \url{tim@menzies.us}}}
    \date{}
    \begin{document}
     \maketitle
    \small


\tableofcontents

\clearpage
\section{Random Predicates\label{Random_Predicates}\index{Random Predicates}}


Lets thrash around a little.

\section{Header\label{Header}\index{Header}}
\subsection{Flags */\label{Flags_}\index{Flags */}}
\begin{verbatim}
 :- arithmetic_function(inf/0).
 :- arithmetic_function(rand/0).
 :- arithmetic_function(rand/2).
 :- arithmetic_function(rand/3).
 :- arithmetic_function(normal/2).
 :- arithmetic_function(beta/1). 
 :- arithmetic_function(gamma/2).  /*
\end{verbatim}
\section{Any  Random Solution  */\label{Any_Random_Solution_}\index{Any  Random Solution  */}}
\begin{verbatim}
 any(X) :- 
        setof(One,X^any1(X,One),All), 
        member(_/X,All).
\end{verbatim}
\begin{verbatim}
 any1(X,Score/X) :- 
        X, 
        Score is rand. /*
\end{verbatim}
\section{Best\label{Best}\index{Best}}


Same code as above, but with some assessment criteria thrown in. */

\begin{verbatim}
 best(X) :- 
        setof(One,X^best1(X,One),All0), 
        beam(X,All0,All), 
        member(_/X,All).
\end{verbatim}
\begin{verbatim}
 best1(X,Score/X) :- 
        X, 
        score(X,Score0), 
        bound(X,Score0), 
        Score is -1 * Score0. /*
\end{verbatim}


If we have no knowledge of \texttt{X}, give it a random number. */

\begin{verbatim}
 score(_,N) :- 
        N is  rand(2147483647). /*
\end{verbatim}


If we have knowledge of minimum values for a score,  test it here. */

\begin{verbatim}
 bound(_,Score) :- 
        Score > 0. /*
\end{verbatim}


Sometimes, we may just want to select the top \textit{N} values:
makes this a beam search. The default
case is that there is no selection knowledge. */

\begin{verbatim}
 beam(_,L,L).  /*
\end{verbatim}
\section{Rand  */\label{Rand_}\index{Rand  */}}
\begin{verbatim}
 rand(X)         :- X is random(inf)/inf.
 rand(Min,Max,X) :-  X is Min + (Max-Min)*rand. /*
\end{verbatim}
\subsection{Beta */\label{Beta_}\index{Beta */}}
\begin{verbatim}
 rand(Min,Max,B,X) :-
    X is Min + (Max-Min)*beta(B).
\end{verbatim}
\begin{verbatim}
 beta(B,X) :- beta1(B,X),!.
 beta(B,X) :- beta1(0.5,X),print(user,badBeta(B)).
\end{verbatim}
\begin{verbatim}
 beta1(0,0).
 beta1(0.1,X)  :- X is 1- rand^(1/9).
 beta1(0.20,X) :- X is 1- rand^0.25.
 beta1(0.25,X) :- X is 1- rand^0.33.
 beta1(0.33,X) :- X is 1- rand^0.5.
 beta1(0.4,X)  :- X is 1- rand^0.67.
 beta1(0.50,X) :- X is rand.
 beta1(0.60,X) :- X is rand^0.67.
 beta1(0.67,X) :- X is rand^0.5.
 beta1(0.75,X) :- X is rand^0.33.
 beta1(0.80,X) :- X is rand^0.25.
 beta1(0.9,X)  :- X is rand^(1/9).
 beta1(1,1). /*
\end{verbatim}
\subsection{Normal  */\label{Normal_}\index{Normal  */}}
\begin{verbatim}
 normal(M,S,N) :- box_muller(M,S,N).
\end{verbatim}
\begin{verbatim}
 box_muller(M,S,N) :-
    wloop(W0,X),
    W is sqrt((-2.0 * log(W0))/W0),
    Y1 is X * W,
    N is M + Y1*S.
\end{verbatim}
\begin{verbatim}
 wloop(W,X) :-
    X1 is 2.0 * rand - 1,
    X2 is 2.0 * rand - 1,
    W0 is X1*X1 + X2*X2,
    (W0  >= 1.0 -> wloop(W,X) ; W0=W, X = X1). /*
\end{verbatim}
\subsection{Gamma\label{Gamma}\index{Gamma}}


(Only works for integer Alphas.)



A standard random \textit{gamma} distribution has 
\textit{mean=alpha/beta}.
The \textit{alpha} value is the ``spread'' of the 
distribution and controls the
clustering of  the distribution
around the mean. As
\textit{alpha} increases,
the \textit{gamma} distribution evens out to become 
more evenly-distributed about the mean. That is, for large
\textit{alpha} (i.e. above 20), \textit{gamma} can be modeled as a noraml
function.
The standard \textit{alpha,beta} terminology can be confusing
to some audiences. Hence, I define a
(slightly) more-intuitive \textit{gamma}
distribution where:



\textit{myGamma(mean,alpha) = standardGamma(alpha,alpha/mean)} */

\begin{verbatim}
 gamma(Mean,Alpha,Out) :-
    Beta is Mean/Alpha,
    (Alpha > 20
    ->  Mean is Alpha * Beta,
        Sd is sqrt(Alpha*Beta*Beta),
        Out is normal(Mean,Sd)
    ;   gamma(Alpha,Beta,0,Out)).
\end{verbatim}
\begin{verbatim}
 gamma(0,_,X,X) :- !.
 gamma(Alpha,Beta, In, Gamma) :-
    Temp is In + ( -1 * Beta * log(1-rand)),
    Alpha1 is Alpha - 1,
    gamma(Alpha1,Beta,Temp,Gamma). /*
\end{verbatim}
\subsection{Random Strings */\label{Random_Strings_}\index{Random Strings */}}
\begin{verbatim}
 rstring(_,X)  :- nonvar(X),!.
 rstring(A,X)  :- gensym(A,X). /*
\end{verbatim}
\subsection{Random Symbols */\label{Random_Symbols_}\index{Random Symbols */}}
\begin{verbatim}
 rsym(X) :- rsym(g,X).
\end{verbatim}
\begin{verbatim}
 rsym(_,X)  :- nonvar(X),!, atom(X).
 rsym(A,X)  :- gensym(A,X). /*
\end{verbatim}
\subsection{Random Members of a List */\label{Random_Members_of_a_List_}\index{Random Members of a List */}}
\begin{verbatim}
 rin(X,L) :- number(X),!, member(Y,L), X =:= Y. 
 rin(X,L) :- nonvar(X),!, member(X,L). 
 rin(X,L) :- any(member(X,L)).  /*
\end{verbatim}
\subsection{Random Numeric Taken From Some Range\label{Random_Numeric_Taken_From_Some_Range}\index{Random Numeric Taken From Some Range}}


The default case is that we step from some \texttt{Min}
to \texttt{Max} number in incremets of one. */

\begin{verbatim}
 rin(Min,Max,X) :- rin(Min,Max,1,X). /*
\end{verbatim}


The usual case is that we step from some \texttt{Min}
to \texttt{Max} number in incremets of \texttt{I}. */

\begin{verbatim}
 rin(Min,Max,_,X) :- nonvar(X),!, number(X),Min =< X, X =< Max.
 rin(Min,Max,I,X) :- any(rin1(Min,Max,I,X)).
\end{verbatim}
\begin{verbatim}
 rin1(Min,Max,I,X) :-
        Steps is integer((Max-Min)/I),
        between(0,Steps,Step), 
        Num is Min + Step*I,
        X is min(Num,Max).  /*
\end{verbatim}
\section{Footer\label{Footer}\index{Footer}}
\subsection{Start-ups */\label{Start-ups_}\index{Start-ups */}}
\begin{verbatim}
 :- current_prolog_flag(max_integer,X), 
    X1 is X - 1, 
    retractall(inf(_)),
    assert(inf(X1)).
\end{verbatim}
\printindex

\end{document}
