\documentclass[twocolumn,global]{svjour}
\usepackage{tex4log}
\journalname{Submitted to WVU Knowledge Engineering}
\date{Wp ref: tim/src/pl/cocomo/cocomo \today}
\newcommand{\ME}{POCOMO}
\begin{document}
\title{\ME: A Prolog-based Software Cost Estimation Tool}
\author{Tim Menzies\inst{1}, Santa Clause\inst{2}}
\institute{Lane Department of Computer Science,
       University of West Virginia,
       PO Box 6109, Morgantown,
       WV, 26506-6109, USA;\\
       \url{http://tim.menzies.us};
       \email{tim@menzies.us} \and
 Artic Software Systems}
\maketitle
\thispagestyle{empty}\pagestyle{plain}
 \begin{abstract}
COCOMO is a software effort estimation tool.
POCOMO is COCOMO written in Prolog and documented using
{\Tex4Log}.
\end{abstract} 
\setcounter{tocdepth}{4}\tableofcontents\listoffigures
\section{% Introduction
}
 Standard format:
\bi
\item
Shell
\item Knowledge base
\item 
An appendix with acknowledgements, references, and licensing details.
\ei
The shell divides up as follows:
\bi
\item Initializations:
\bi
\item
Operator definitions (must be first).
\item
Flags (these can usually go just before the start-up
actions but, for safety's sake, we place them at the
front).
\item
hooks (into the Prolog reader)
\item
hacks (shameful things we'd rather hide).
\ei
\item
The actual system code.
\item
Library code which, ideally,
should be good for more than just this application).
\item
Start-up code (must be loaded into Prolog last).
\ei 
\section{ Initializations
}
\subsection{ Operators
}
\subsection{ Flags
}
\subsection{ Hooks
}
\subsection{ Hacks
}
\section{ Library
}
\subsection{ Configuration Control }\begin{Verbatim}
[] := []    :- !.
[H0|T0] := [H|T] :- !, H0 := H, T0 := T.
X := Y      :-  option(X,Z),!, Y=Z.
X := _      :-  !, barph(missingOption(X)).
?X          :- atomic(X), X := 1.

set(X=Y) :- retractall(option(X,_)), assert(option(X,Y)).

commandLine :-
	current_prolog_flag(argv, Argv),
	append(_, [--|Args], Argv), !,
	concat_atom(Args, ' ', SingleArg),
	term_to_atom(Term, SingleArg),
	c2l(Term,List),
	forall(member(One,List), set(One)).
commandLine.
\end{Verbatim}
\subsection{ Demo support code
}

Catches the output from some predicate X
and saves it a file X.spy. The command:

{\scriptsize
\begin{verbatim}
\SRC{X.spy}{Caption}
\end{verbatim}}
\noindent
includes the generated file into the {\LaTeX} document.

The code `demos/1` deletes any old output and runs some goal twice: once
to trap it to a file and once to show the results on the screen.  \begin{Verbatim}
demos(G) :-
	sformat(Out,'~w.spy',G),
	(exists_file(Out) -> delete_file(Out) ; true),
	tell(Out),
	format('% output from '':- demos(~w).''\n\n',G),
        T1 is cputime,
	ignore(forall(G,true)),
	T2 is (cputime - T1),
	format('\n% runtime = ~w sec(s)\n',[T2]),
        told,
	format('% output from '':- demos(~w).''\n',G),  
        ignore(forall(G,true)),
	format('\n% runtime = ~w sec(s)',[T2]).
\end{Verbatim}
\subsection{ Dump a Whole File to the Screen}\begin{Verbatim}
chars(F) :- see(F), get_byte(X), ignore(chars1(X)), seen.
    
chars1(-1) :- !.
chars1(X)  :- put(X), get_byte(Y), chars1(Y).
\end{Verbatim}
\subsection{ License }\begin{Verbatim}
hello :-  % \MARK{hello}
	[program,version,copyright,motto,copywho]:=[N,V,Y,M,C],
	format('~s version ~s\n Copyright (C) ~s by ~s\n',
	       [N,V,Y,C]),
	format(' "~s"\n\n~s ~s ',[M,N,V]),
	chars('nowarranty.txt'). % see \tion{nowar}
   
warranty :-
	[program,copyright,copywho]:=[P, Y,C],
	format('~s by ~s\n Copyright (C) ~s\n\n',[P,C,Y]),
	chars('warranty.txt'),nl. % see \tion{war}

conditions :-
	chars('conditions.txt'),nl. % see \tion{cond}
\end{Verbatim}
\subsubsection{ Misc }\begin{Verbatim}
sneak(X) :- load_files(X,[silent(true),if(changed)]).

spit(N1,N2,X) :- (0 is N1 mod N2 -> spit(X) ; true).

spit(X) :- ?verbose,!,write(user,X),flush_output(user).
spit(_).

barph(X) :- format('%W> ~p\n',X),fail.
\end{Verbatim}
\section{ Start up actions}\begin{Verbatim}
:- sneak(
	['defaults.nb' % see \tion{defaults}
        ,'config.nb'   % see \tion{config}
    ]).

:- commandLine.
:- ?verbose -> hello ; true.
:- namesData.
\end{Verbatim}
\acknowledgement{\input{thanks}}
\bibliographystyle{abbrv}{
\footnotesize \bibliography{refs}
\appendix
\input{license}
\end{document}

