\section{Style notes}

\subsection{Prolog Source Code Idioms}

\subsubsection{Preparing Include Files}

Divide your Prolog
code up into lots of little files. If it can't fit into one
column, it is too big. In practice, this means  no file is more than
55 characters wide and 80 lines long.

Predicates can't have long argument lists. Practically speaking, this
often implies that you need DCGs to handle carrying round updatable
state.

Note that the included files will be processed by \LaTeX
in a verbatim mode. The good news is that this
 implies that \LaTeX commands can be added
into the Prolog source code. Such commands can perform useful
tasks such as (e.g.) linking from one figure to another figure.
The bad news is that some source code constructs will be
inappropriately rendered by Prolog. For example !\t! denotes
 a tab character in a string. In \Text2Log,  it gets read as a \LaTeX
 macro with inappropriate side-effects.



\subsubsection{Including Source Code}

Source code should be included into the {\LaTeX} source using the
!\code{app/file}{Caption}! command. This will generate a {\LaTeX}
reference !fig:file! which can be accessed in the usual way using
!\fig{app/file}!. Included source code gets rendered in a box which
includes the name of the file.

\subsubsection{Marking Lines in Source Code}


Within a source code file,
a line can be marked with
!\MARK{X}! where !X! can be
referenced in the \LaTeX source
using any of the following macros.
\bi
\item
!\Line{X}!
renders as  (e.g.) ``line  23''.
\item
!\Where{X}!
renders as  (e.g.) ``line  23 in Figure 5''.
\ei
If these references are at start of sentence, they need a leading
capital letter. So, two more macros are defined:
\bi
\item
!\LINE{X}!
renders as  (e.g.) ``Line  23''.
\item
!\WHERE{X}!
renders as  (e.g.) ``Line  23 in Figure 5''.
\ei

\subsubsection{Referring to Included Files}

Anytime a Prolog program loads a file, it should be followed by a
{\LaTeX} figure reference. For example:
\newcommand{\bslash}{$\backslash$}

\bl
:- [abc]. % see {\bslash}fig\{xyz/abc.pl\}
\el

This way, if you forget to include !abc.pl! , you'll get one of those
``?'' warnings that {\LaTeX} generates (so you'll know if you've
missed anything).

\subsubsection{Including Output From Demonstration Code}

A Prolog idiom in \Tex4Log
is that some source code comes in sets of threes:
\be
\item !X.pl!
source code.
\item
  !Xeg.pl! demonstration code that exercises
!X.pl!.
\item
!Xeg.spy! contains the output generated when !Xeg.pl! was
executed.
\ee

The macro !\peektell{X}! will include !Xeg.pl! and
!Xeg.spy!, if those files exist.

The macro !\showtell{X}! extends !\peektell{X}!.
This macro includes !X.pl! and
!Xeg.pl! and
!Xeg.spy!, if those files exist.



\subsection{\LaTeX~Idioms}

\TeX4LoG includes some useful and succinct \LaTeX macros. This
section describes those macros.

The silliest macro is  !\Tex4Log!
which renders as \Tex4Log.

Also:
\bi
\item
Itemized lists
 can be succinctly started and stopped using\newline !\bi\item...\ei!.
 \item
 Enumerated lists
 can be succinctly started and stopped using \newline
   !\be\item...\ee!.
    \item
  Definition lists can be succinctly started and stopped using \newline
!\bd\item...\ed!.
\ei

A simple code listing environment with small font size is available
via !\begin{LISTING}...\end{LISTING}! or  !\bl...\el!.

In-line verbatim text can be typeset  using pairs of exclamation
marks; e.g. \verb+!text!+ is set as !text!.
If you need an exclamation mark in your text, use the !\exclaim!
macro which expands to ``\exclaim''.

Cited numeric references are sorted and
ranges are shortened to dashes; e.g.
!\cite{men01c,men01b,men01a}! could be rendered
as [1-3].

Long URLs can be typeset using !\url{...}!. The !\url!
macro renders URLs in verbatim font and long strings
may be broken at the ``!/!'' character
(if that improves their rendering).

If figures are labelled with !\label{fig:abc}!,
then they can be referenced using (e.g.) !\fig{abc}!.
This is typeset as (e.g.) Figure~11.

If equations are labelled with !\label{eq:def}!,
then they can be referenced  using (e.g.) !\eq{def}!.
This is typeset as (e.g.) Equation~11.

If sections are labelled with !\label{sec:ghi}!,
then they can be referenced  using (e.g.) !\tion{ghi}!.
This is typeset as (e.g.) \S~2.

\subsection{Prolog Idioms}

You application will comprise one main file that loads everything
else. Usually this list is loaded silently:

\bl
:- load_files(
       [myapp/file1 % see {\bslash}fig\{myapp/file1.pl\}
       ,myapp/file2 % see {\bslash}fig\{myapp/file2.pl\}
       ],
       [silent(true) % no stray text on screen
       ,if(changed)  % load files only once
       ]).
\el

The load order can vary by often includes: \bd \item[operators] which
tells the Prolog parser how to handle the psuedo-English constructs
in your code. \item[hooks] that define your !term_expansion! and
!goal_expansion! rules. \item[hacks] that define some patches to our
Prolog system \ \ed

\bi \item {\em Many library files} that set things up; \item {\em
Some core files} that combine the libraries to perform
 some interesting task;
 \item
 {\em Some  example files} that
show off the core code. \ei For every core file !core.pl!, there
should be a !coreeg.pl! example file containing   a !coreeg!
predicate that does the showing off. !coreeg.pl! should have  the
following structure:

\bl
:- [lib1 % see {\bslash}fig\{lib1.pl\}
   ,lib2 % see {\bslash}fig\{lib2.pl\}
   ].
...
ccode :- doSomething.
:- demos(coreeg). % call to the core code that traps output
\el

A file with this structure will generate a file !coreeg.out! as a
side effect of loading !coreeg.pl!. In order to include the core,
example, and output files, the macro !\showtell{core}! is useful. The
call !\showtell{core}! generates three figures: one for !coreeg.pl!,
one for !coreeg.out! and one for !core.pl!.

It is good practice to reference the core, example, and output
files in the main file; e.g.

\bl
:-  [abc]. % see {\bslash}fig\{abc.pl\},  {\bslash}fig\{abceg.pl\} &  {\bslash}fig\{abceg.out\}
\el



Never use a backslash such as !\X! so: \bi \item Use !not(X)! instead
of !\+ X! \item Use !nl! instead of  !\n! . \ei

Never use !{X}! since {\LaTeX} can't print that. So: \bi \item Define
!goal_expansion/2! to repair over-zealous DCG expansions (see
\fig{hacks.pl}). \item Use !code(X)! to denote code that should be
not be DCG-expanded. Then add the following item to !hooks.pl!

\begin{LISTING}
goal_expansion(call(A,B,B),A).
\end{LISTING}

\ei
\subsection{Other Requirements}

\subsubsection{Peekers}

A {\em peeker} is a set of predicate sthat  ``looks before it
leaps''. The idea is that before doing anything on !X!, the code
peeks at !X! and then uses what it sees to determine what actions to
perform. Peekers, as used in this code, are filters that convert !X!
to !Y!. Their general form is three predicates: !p,p0,p1!:

\begin{LISTING}
p(X,Y) :- once(p0(X,Z)), p1(Z,Y).

p0(X,          barph(X) ) :- var(X). % usual top peeker
p0(X,  barph(number(X)) ) :- number(X). % e.g.
p0((X,Y),         (X,Y) ). ...
p0(X,       standard(X) ). % some final default action

p1(barph(X),  _)   :- print(bad(X)), fail.
p1(standard(X),Y)  :- Y is X % e.g.
p1((X0,Y0), (X,Y)) :- p(X0,X), p(Y0,Y). ...
\end{LISTING}

The !p0! predicate classifies the incoming structure into one of $N$
terms. One !p1! predicate is written for each such term. If !p1!
recurses, then the recursive call is back to !p!.

Peekers have several  useful features: \bi \item Fewer/no cuts in the
code. \item Stepping through the !spy! outputs over for a peeker is
usually easier since only one choice for !p1! is ever shown and the
!p0! processing is often simple enough that the programmer can just
skip it. \item Often there are less !p1! predicates than !p0!
predicates since !p0! might map different incoming terms to the same
functor (e.g., in the above, while there are two situations where we
want to !barph!, we only needed to write !barph! once. \item If !p0!
is constrained to key just on features available when the source code
is loaded for the first time, then the peeker can be optimized as
follows. At load time, pre-compute and cache all the !p0! results on
the loaded code. Then, at runtime, replace all the !p0! with
!p0(X,X)! and all that lookup computation disappears. \ei
