\documentclass[twocolumn,global]{svjour}
\usepackage{tex4log}
\journalname{Submitted to WVU Knowledge Engineering}
\date{Wp ref: tim/src/pl/nb/nb \today}
\newcommand{\ME}{NB}
\begin{document}
\title{\ME: A Prolog-based Naive Bayes Classifier}
\author{Tim Menzies\inst{1}, Santa Clause\inst{2}}
\institute{Lane Department of Computer Science,
       University of West Virginia,
       PO Box 6109, Morgantown,
       WV, 26506-6109, USA;\\
       \url{http://tim.menzies.us};
       \email{tim@menzies.us} \and
 Artic Software Systems}
\maketitle
\thispagestyle{empty}\pagestyle{plain}
 \begin{abstract}
{\Tex4Log} is a simple
set of macros written in {\LaTeX} allowing
for a simple documentation
scheme for Prolog.
\end{abstract}
\setcounter{tocdepth}{4}
\tableofcontents
\listoffigures
\newpage

\section{ Introductions
}
 The following package is a SWI-Prolog system~\cite{swiprolog}.
Prolog is a useful language for rapdily building
systems~\cite{der96,me89zb,bratko01,ok90,sterling94}.
This {\Tex4Log}
description  follows my standard(ish) pattern:
\bi
\item
Shell
\item Knowledge base
\item
An appendix with acknowledgements, references, and licensing details.
\ei
The shell divides up as follows:
\bi
\item Initializations:
\bi
\item
Operator definitions (must be first).
\item
Flags (these can usually go just before the start-up
actions but, for safety's sake, we place them at the
front).
\item
hooks (into the Prolog reader)
\item
hacks (shameful things we'd rather hide).
\ei
\item
The actual system code.
\item
Library code which, ideally,
should be good for more than just this application).
\item
Start-up code (must be loaded into Prolog last).
\ei

\section{ Initializations
}
\subsection{ Operators }\begin{Verbatim}
:- op(1001,xfx, the).
:- op(1001,xfx, a  ).
:- op(999 ,fx,  *  ).
:- op(700,xfx, :=).
:- op(1 ,fx, (?) ).
:- op(1 ,fx, (!) ).
\end{Verbatim}
\subsection{ Flags }\begin{Verbatim}
:- multifile     option/2, meta/8, oo/5.
:- dynamic       option/2, meta/8, oo/5.
:- discontiguous option/2, meta/8, oo/5.
\end{Verbatim}
\subsection{ Hooks }\begin{Verbatim}
term_expansion((X the Rel),Z) :- dd(Rel,X,Z).
term_expansion((X a   Rel),Z) :- eg(Rel,X,Z).
term_expansion(A=B,       []) :- set(A=B).
\end{Verbatim}
\subsection{ Hacks
}
 Shown here are some dark secrets of the Prolog
wizards. If you don't yet understand
the following code,
then you don't need to know it. Trust us,
we are knowledge engineers.  \begin{Verbatim}
goal_expansion(is(A,B,C,C),            is(A,B)).
goal_expansion(=(A,B,C,C),              =(A,B)).
goal_expansion(>=(A,B,C,C),            >=(A,B)).
goal_expansion(>(A,B,C,C),              >(A,B)).
goal_expansion(<(A,B,C,C),              <(A,B)).
goal_expansion(<=(A,B,C,C),            <=(A,B)).
\end{Verbatim}
\section{ Library code
}
\subsection{ Does a goals have 1 matching clause? }\begin{Verbatim}
singleton(X) :-
    Sym='$sym',
    flag(Sym,_,0),
    \+ doubleton(Sym,X),
    flag(Sym,1,1).

doubleton(Sym,X) :- clause(X,_),flag(Sym,N,N+1),N > 1.

clause1(X,Y) :- singleton(X), clause(X,Y).
\end{Verbatim}
\subsection{ Configuration Control }\begin{Verbatim}
[] := []    :- !.
[H0|T0] := [H|T] :- !, H0 := H, T0 := T.
X := Y      :-  option(X,Z),!, Y=Z.
X := _      :-  !, barph(missingOption(X)).
?X          :- atomic(X), X := 1.

set(X=Y) :- retractall(option(X,_)), assert(option(X,Y)).

commandLine :-
    current_prolog_flag(argv, Argv),
    append(_, [--|Args], Argv), !,
    concat_atom(Args, ' ', SingleArg),
    term_to_atom(Term, SingleArg),
    c2l(Term,List),
    forall(member(One,List), set(One)).
commandLine.
\end{Verbatim}
\subsection{ Demo support code
}

Catches the output from some predicate X
and saves it a file X.spy. The command:

{\scriptsize
\begin{verbatim}
\SRC{X.spy}{Caption}
\end{verbatim}}
\noindent
includes the generated file into the {\LaTeX} document.

The code `demos/1` deletes any old output and runs some goal twice: once
to trap it to a file and once to show the results on the screen.  \begin{Verbatim}
demos(G) :-
    sformat(Out,'~w.spy',G),
    (exists_file(Out) -> delete_file(Out) ; true),
    tell(Out),
    format('% output from '':- demos(~w).''\n\n',G),
        T1 is cputime,
    ignore(forall(G,true)),
    T2 is (cputime - T1),
    format('\n% runtime = ~w sec(s)\n',[T2]),
        told,
    format('% output from '':- demos(~w).''\n',G),
        ignore(forall(G,true)),
    format('\n% runtime = ~w sec(s)',[T2]).
\end{Verbatim}
\subsection{ Dump a Whole File to the Screen}\begin{Verbatim}
chars(F) :- see(F), get_byte(X), ignore(chars1(X)), seen.

chars1(-1) :- !.
chars1(X)  :- put(X), get_byte(Y), chars1(Y).
\end{Verbatim}
\subsection{ License }\begin{Verbatim}
hello :-  % \MARK{hello}
    [program,version,copyright,motto,copywho]:=[N,V,Y,M,C],
    format('~s version ~s\n Copyright (C) ~s by ~s\n',
           [N,V,Y,C]),
    format(' "~s"\n\n~s ~s ',[M,N,V]),
    chars('nowarranty.txt'). % see \tion{nowar}

warranty :-
    [program,copyright,copywho]:=[P, Y,C],
    format('~s by ~s\n Copyright (C) ~s\n\n',[P,C,Y]),
    chars('warranty.txt'),nl. % see \tion{war}

conditions :-
    chars('conditions.txt'),nl. % see \tion{cond}
\end{Verbatim}
\subsection{ Miscalleanous
}
 Pretty print a list of terms.  \begin{Verbatim}
portrays(L) :- portrays(L,_,_).

portrays([],_,_).
portrays([H|T],F0,A0) :-
    functor(H,F,A),
    (F0=F,A0=A
        -> portray_clause(H),
           portrays(T,F0,A0)
         ; nl,portray_clause(H),
           portrays(T,F,A)).
\end{Verbatim}
 Other stuff.  \begin{Verbatim}
c2l((X,Y),[X|Z]) :- !,c2l(Y,Z).
c2l(X,[X]).

l2c([W,X|Y],(W,Z)) :- l2c([X|Y],Z).
l2c([X],X).

mostC2l((X,Y),[X|Z]) :- !,mostC2l(Y,Z).
mostC2l(_,[]).

sneak(X) :- load_files(X,[silent(true),if(changed)]).

spit(N1,N2,X) :- (0 is N1 mod N2 -> spit(X) ; true).

spit(X) :- ?verbose,!,write(user,X),flush_output(user).
spit(_).

barph(X) :- format('%W> ~p\n',X),fail.
\end{Verbatim}
\subsection{ Wrapper }\begin{Verbatim}
wrapper(X,Out) :-
    wrap(X,Before,[],After,[],Goal),
    append(Before,[Goal|After],Temp),
    l2c(Temp,Out).

wrap(X,B0,B,A0,A,Y) :-
    once(wrap0(X,Z)),
    wrap1(Z,B0,B,A0,A,Y).

wrap0(X,        leaf(X) ) :- var(X).
wrap0(X,        leaf(X) ) :- atomic(X).
wrap0([],    leaf(true) ).
wrap0([H|T],      [H|T] ).
wrap0(?X,            ?X ).
wrap0(!X,            !X ).
wrap0(X,        term(X) ).

wrap1(leaf(X),     B, B, A, A, X).
wrap1([H0|T0],     B0,B, A0,A, [H|T]) :-
    wrap(H0,   B0,B1,A0,A1,H),
    wrap(T0,   B1,B, A1,A, T).
wrap1(term(X),     B0,B, A0,A, Y) :-
    X =.. L0,
    wrap(L0,   B0,B,A0,A,L),
    Y =.. L.
wrap1(?X, [ooo(X,Y,Y)|B],B,A,A,Y).
wrap1(!X, B,B,[ooo(X,_,Y)|A],A,Y).

%interpreter
ooo(X,Y0,Y,T0,T) :- oo(T0,X,Y0,Y,T).
%optimizer
goal_expansion(ooo(X,Y,Z,T0,T), oo(T0,X,Y,Z,T)).
\end{Verbatim}
\subsection{ Accessors
}
\subsubsection{ Usage }\begin{Verbatim}
o(Com,X) :- o(Com,X,_).

o([],X,X).
o([H|T],X,Y)  :- o(H,X,Z), o(T,Z,Y).
o(F >= T,X,X) :- oo(X,F,V,V,X), V >= T.
o(F < T,X,X)  :- oo(X,F,V,V,X), V <  T.
o(!F=V,X,Y)   :- !,oo(X,F,_,V,Y).
o(-F,X,Y)     :- oo(X,F,V0,V,Y), V is V0 - 1.
o(+F,X,Y)     :- oo(X,F,V0,V,Y), V is V0 + 1.
o(F+V,X,Y)    :- oo(X,F,L,[V|L],Y).
o(F is Z,X,Y) :- oo(X,F,_,V,Y), V is Z.
o(F=V,X,X)    :- oo(X,F,V,V,X).

goal_expansion(o(X,Y),o(X,Y,_)).

goal_expansion(o([H],X,Y), o(H,X,Y)) :-
goal_expansion(o([H1,H2|T],X,Y),
           (o(H1,X,Z),o([H2|T],Z,Y))).

goal_expansion(o(X,Y,Z),Body) :-
    clause1(o(X,Y,Z),Body).

goal_expansion(oo(X,F,V0,V,Y),Body) :-
    clause1(oo(X,F,V0,V,Y),Body).

% XXX screw up here
goal_expansion(*(X,Y,Y),W) :- wrapper(X,W).
\end{Verbatim}
\subsubsection{ Field details
}
 Each field has annotations that indicate:
\be
\item If Prolog should index on a particular field.
\item A fields name and default falue.
\item Some rule that defines the
valid range of a field.
\ee  \begin{Verbatim}
detail(+X:R=D,1,      X,   R,   D).
detail(+X:R,  1,      X,   R,   _).
detail(+X=D,  1,      X,   any, D).
detail(+X,    1,      X,   any, _).
detail(X:R=D, 0,      X,   R,   D).
detail(X:R,   0,      X,   R,   _).
detail(X=D,   0,      X,   any, D).
detail(X,     0,      X,   any, _).
\end{Verbatim}
These details are stored in a `dd0/8` fact which we
can manipulate as follows:   \begin{Verbatim}
oo(meta(A,B,C,D,E,F,G,H),X,Y,Z,Out) :- % \MARK{meta/5}
    dd0(X,Y,Z,meta(A,B,C,D,E,F,G,H),Out).

dd0(rel0, X,Y,meta(X,B,C,D,E,F,G,H),meta(Y,B,C,D,E,F,G,H)).
dd0(arity,X,Y,meta(A,X,C,D,E,F,G,H),meta(A,Y,C,D,E,F,G,H)).
dd0(rel,  X,Y,meta(A,B,X,D,E,F,G,H),meta(A,B,Y,D,E,F,G,H)).
dd0(index,X,Y,meta(A,B,C,X,E,F,G,H),meta(A,B,C,Y,E,F,G,H)).
dd0(names,X,Y,meta(A,B,C,D,X,F,G,H),meta(A,B,C,D,Y,F,G,H)).
dd0(rules,X,Y,meta(A,B,C,D,E,X,G,H),meta(A,B,C,D,E,Y,G,H)).
dd0(inits,X,Y,meta(A,B,C,D,E,F,X,H),meta(A,B,C,D,E,F,Y,H)).
dd0(blank,X,Y,meta(A,B,C,D,E,F,G,X),meta(A,B,C,D,E,F,G,Y)).
\end{Verbatim}
\subsubsection{ Core engine
}
 Code for being able to  access and changed
named fields within a term.  \begin{Verbatim}
dd(Rel0,FieldsC,All) :-
    dd1s(new(Rel0,FieldsC),Meta),
    bagof(One, Meta^dd2(Meta,One), All).

dd1s(new(Rel0,Fields0),Out) :-
    atom_concat(Rel0,'_',Rel),
    c2l(Fields0,Fields1),
    reverse([+id_|Fields1], [_|Fields]),
    length(Fields,Arity),
    functor(Blank,Rel0,Arity),
    dd1(Fields,
              meta(Rel0,Arity,Rel,[],[],[],[],Blank),
           Out).

dd1([],X,X).
dd1([H|T]) -->
    {detail(H,I,N,R,D)},
    o([index+I, names+N,rules+R,inits+D]),
    dd1(T).
\end{Verbatim}
\subsubsection{ dd2/2
}
 Reset the counter for this relation to zero.  \begin{Verbatim}
dd2(X,(:- flag(Rel0,_,0))) :- o(rel0=Rel0,X).
\end{Verbatim}
 Create an index on the indexed fields.  \begin{Verbatim}
dd2(X,(:- index(Index))) :-
    o([rel0=Rel0,index=Index0],X),
    Index =.. [Rel0|Index0].
\end{Verbatim}
 Note that assertions in this relation may or may not
exist at a particular time.  \begin{Verbatim}
dd2(X,(:- dynamic R /A)) :- o([rel0=R,arity=A],X).
\end{Verbatim}
 Automatically generate arity five accessors, just like
the ones written manually at \Line{meta/5}.
\begin{Verbatim}
dd2(Meta,Out) :-
    o([names=Names,rel0=Rel0,rel=Rel,arity=Arity],Meta),
    nth0(Pos,Names,Name),
    length(Before,Pos),
    functor(Term0, Rel0,Arity), Term0 =.. [_|L0],
    functor(Term,  Rel0,Arity), Term  =.. [_|L1],
    append(Before,[Old|After],L0),
    append(Before,[New|After],L1),
    Out =.. [Rel,Name,Old,New,Term0,Term].
\end{Verbatim}
 Define a {\em bridge} predicate that calls the
arity five accessor that is relevant to a particular term.  \begin{Verbatim}
dd2(X,(oo(T0,Com,V0,V,T) :- Body)) :-
    o([rel0=Rel0,rel=Rel,arity=Arity],X),
    functor(T0,Rel0,Arity),
    Body =.. [Rel,Com,V0,V,T0,T].
\end{Verbatim}
 Finish up the meta-level fact.  \begin{Verbatim}
dd2(X,Y) :-
    o([rel0=Rel0,inits=Inits0],X),
    Inits =.. [Rel0|Inits0],
    o(!inits=Inits,X,Y).
\end{Verbatim}
 Write our terms with names fields very succinctly.  \begin{Verbatim}
dd2(X,(portray(Term) :- write(Rel/ Arity))) :-
    o([rel0=Rel,arity=Arity],X),
    functor(Term,Rel,Arity).
\end{Verbatim}
 Singlton accessors with arity 3 can be expanded to accessor 5   \begin{Verbatim}
dd2(X,(goal_expansion(H,Body)  :- clause1(H,Body))) :-
    o(rel=Rel,X),
    H =.. [Rel,_,_,_].
\end{Verbatim}
 Singleton accessors with arity 5 can be evaluated,
    then replaced with `true`.  \begin{Verbatim}
dd2(X,(goal_expansion(H,true) :- clause1(H,true))) :-
    o(rel=Rel,X),
    functor(H,Rel,5).
\end{Verbatim}
\subsubsection{ Demos }\begin{Verbatim}
egdd :- % \see{egdd.spy}
    expand_term(
       (+deptNo : num
        ,name   : [x,y,z]
        ,+age   : num = 1
        ,are the eg),
       X),
    portrays(X).
\end{Verbatim}
 \SRC{egdd.spy}{}

\subsubsection{ Generating an instance
}
 see \Line{d}  \begin{Verbatim}
eg(Rel,FieldsC,Datum) :-
    mostC2l(FieldsC,FieldsL),
    flag(Rel,M,M+1),
    N is M + 1,
    spit(N,50,+),
    Datum =.. [Rel,N|FieldsL].
\end{Verbatim}
\subsection{ Statistics }\begin{Verbatim}
stale=0, mean:num=0, sd:num=0,sum:num=0
,sumSquared:num=0, are the gaussian.

gaussian_(add(X)) -->
    * !stale=1,
        * !sum is ?sum+X,
    * !sumSquared is ?sumSquared + X*X.

stale, bins=[], total: num=0, are the histogram.
\end{Verbatim}
\section{ The NB system
}
  If an application stands on some library,
    then there is some hope that the library
    may be useful elsewhere even if the application
is not. So, the first rule of Timm: a good application is an empty
application; i.e. is just a place where
supposedly reusable components work together for a while.

\subsection{ Loading }\begin{Verbatim}
namesData :- names,data.

names :- relation:=R, names(R).
data  :- relation:=R, data(R).

names(X) :- atom_concat(X,'.names',Y), [Y].
data(X)  :- atom_concat(X,'.data', Y), [Y].
\end{Verbatim}


\section{ Start up actions}\begin{Verbatim}
:- sneak(
    ['defaults.nb' % see \tion{defaults}
        ,'config.nb'   % see \tion{config}
    ]).

:- commandLine.
:- ?verbose -> hello ; true.
:- namesData.
\end{Verbatim}
\acknowledgement{\input{thanks}}
\bibliographystyle{abbrv}{
\footnotesize \bibliography{refs}
\appendix
\input{license}
\end{document}
