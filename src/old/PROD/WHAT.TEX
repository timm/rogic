\section{What?}

\begin{figure}[!b]
\begin{center}
\includegraphics[width=3.2in]{prod.eps}
\end{center}
\caption[Applications within the current release of
{\PROD}]{Applications within the current release of {\PROD}. $Y
\rightarrow X$ indicates that $Y$ has to first load
$X$.}\label{fig:components}
\end{figure}


 {\PROD} can be used to
  {\em document} or {\em deliver}
 a Prolog application:

 \bd
 \item[Delivery:]
 The current {\PROD} distribution comes with a set of standard Prolog applications
  shown
in \fig{components}. These programs are written in
the {\em {\PROD}-compatible file convention} (described in \tion{prodcompat})
which
 simplifies using
Prolog code from different programmers. {\PROD} files are valid
Prolog code that can be loaded into a Prolog interpreter, without
modification.
 \item[Documentation:]A {\PROD} file also
  contains{\LaTeX} commands inside Prolog's comment
characters; i.e. on a line after the {\tt \%} character or between
 {$\backslash$}{\tt *}...{\tt *}{$\backslash$} characters.
 That is, as a programmer
 writes their code  they can add in comments which,
 subsequently, can be typeset.
\ed

The typeset form of a {\PROD} document looks just like this
document and includes a table of contents; a list of figures;
automatic numbering of sections, figures, and citations. Also, all
the Prolog code is displayed as verbatim text (in a {\tt
typewriter font}).

{\PROD} is distributed under the GNU General Public License. Every
{\PROD} document automatically includes that license as part of
its appendix.

\begin{figure}
{\scriptsize \begin{verbatim}
 1 /*\documentclass[twocolumn,global]{svjour}
 2 \usepackage{prod}\begin{document}
 3
 4 \theprogram{NAME}
 5 \thetocdepth{2}
 6 \thepapers{refs}
 7 \thewp{PATHNAME}
 8 \thetitle{TITLE}
 9 \theauthor{AUTHOR1\inst{1},AUTHOR2\inst{2}}
10 \theinstitute{WHERE AUTHOR1 WORKS;\\
11           \url{author1@email1.com},
12           \url{http://where.to.find.author1}
13              \and
14              WHERE AUTHOR2 WORKS}
15 \thereference{WVU, CSEE, AI lab memo \#3. Available from
16         \url{http://tim.menzies.com.pdf/03prod0.pdf}}
17 \theacknowledgement{ACKNOWLEDGEMENTS}
18 \theabstract{ONE PARAGRAPH SUMMARY}
19 */
20
21 %%%% SECTION1 heading
22 /*
23 BODY OF DOCUMENT WITH A REFERNCE~\cite{swiprolog}.
24 */
25 %\input{prod0a}
26 We can include text like that shown in \fig{prod0a.tex}.
27 \SRC{prod0a.tex}{A sample include file.}
28
29 /* Some text to be typeset
30 */
31 %%% SECTION2 heading
32 /* Some text to be typeset
33 */
34 %% SECTION3 heading
35 /* Some text to be typeset
36 */
37 %%%% SECTION1 heading %<
38 somePrologCode :-
39     subGoal1,
40     subGoal2.
41 %>
42 /* Some text between code.
43 */
44 %<
45 someMorePrologCode :-
46     subGoal1,
47     subGoal2.
48 %>
49 /*
50 \theend
51 \end{document}
52 */
\end{verbatim}}
\caption[{\tt prod0.pl}]{{\tt prod0.pl}, a sample {\PROD}
file.}\label{fig:prod0sample}
\end{figure}

\fig{proddocuments}, at the end of this document, lists other
documents relating to {\PROD}.

\subsection{Inside a {\PROD} file}

\fig{prod0sample} shows a sample {\PROD} file. When typeset,
{\LaTeX} converts this document to the PDF file shown at
\url{http://tim.menzies.com/pdf/prod0.pdf}. This file contains a
{\em header}, a {\em preamble}, a {\em body}, and a {\em footer}.

\subsubsection{The header and footer}

A {\PROD} file begins with a standard {\em header}:

{\scriptsize \begin{verbatim}
/*\documentclass[twocolumn,global]{svjour}
\usepackage{prod}\begin{document}
\end{verbatim}}

\noindent which starts a {\LaTeX} document and loads the {\tt
prod.sty} style file. The file also ends with a standard {\em
footer}:

{\scriptsize \begin{verbatim}
/*
\theend
\end{document}
*/
\end{verbatim}}





\subsubsection{The Premable}

In between the footer and the header there is a
{\em preamble} and a {\em body}.
The preamble defines
certain key parameters of the file using the following commands.
 For a detailed example of the use of these commands,
see \fig{samplepreamble}.

\bd
\item[{\tt $\backslash$theprogram\{NAME\}}]: Defines the {\tt NAME} of the program
being described. I use very short names for my programs (less than 3 letters).
\item[{\tt $\backslash$thetocdepth\{LEVEL\}}]: Controls how detailed is the table of contents.
A {\tt LEVEL=N}, the table of contents only includes down to level {\tt N}. For very short
tables of contents, use {\tt N=1}.
\item[{\tt $\backslash$theref\{FILE\}}]:
Shows the location of the file {\tt FILE.bib} which
contains
the citations
for this file.
For those not familiar with {\LaTeX}'s citation system, \fig{bib} shows a sample
of the {\tt refs.bib} database.
\item[{\tt $\backslash$thewp\{PATHNAME\}}]: Shows where to find the source code file
for this document.
\item[{\tt $\backslash$thetitle\{TITLE\}}]: Defines the {\tt TITLE} of the paper.
\item[{\tt $\backslash$theauthor\{AUTHOR1 $\backslash$inst\{1\},
                                   AUTHOR2{\tt $\backslash$inst\{2\}}\}}]:
Defines the {\tt AUTHOR}s and maps those authors to their {\tt INSTITUTIONS}.
\item[{\tt $\backslash$theinstitute\{WORK PLACE\}}]: Defines where the {\tt AUTHOR}s
work.
Multiple {\tt INSTITUTIONS} are separated by ``$\backslash${\tt and}''.
\item[{\tt $\backslash$thereference\{REFERENCE\}}]: Where this paper appears and where it
can it be downloaded from.
\item[{\tt $\backslash$theacknowledgement\{ACKNOWLEDGEMENTS\}}]: Credit given where credit is due.
\item[{\tt $\backslash$theabstract\{ONE PARAGRAPH SUMMARY\}}]: A short summary of the paper.
\ed

Some of the above commands can be entered in a different order
but, for safety's sake, it is best to use the above ordering for
the preamble.

\begin{figure}
{\scriptsize \begin{verbatim}
\theprogram{PROD1}
\thetocdepth{2}
\therefs{refs}
\thewp{~menzies/src/pl/prod/prod0.tex}

\thetitle{An example of the {\PROD}\newline Prolog
             delivery and documentation system}

\theauthor{Tim Menzies\inst{1}, Sant A. Clause\inst{2}}

\theinstitute{Lane Department of Computer Science,
             University of West Virginia,
             PO Box 6109, Morgantown,
             WV, 26506-6109, USA;\\
             \url{http://tim.menzies.us};
             \url{tim@menzies.us}
             \and
            Artic Software Solutions:
              no ice cube too small, no iceberg too big;\\
             \url{http://north.pole/~santac};
             \url{santa@north.pole}
}

\thereference{WVU, CSEE, AI lab memo \#1, 2003.
            Available from
            \url{http://tim.menzies.com/pdf/03prod1.pdf}
}

\theacknowledgement{This research was conducted at

            West Virginia University under NASA
            contract NCC2-0979.
            The work was sponsored by the NASA
            Office of Safety and Mission Assurance
            under the Software Assurance Research
            Program led by the NASA IV\&V Facility.
            Reference herein to any specific
            commercial product, process, or
            service by  trade name, trademark,
            manufacturer, or otherwise, does not
            constitute or imply its endorsement
            by the United States Government.
}

\theabstract{This document is a minimal example of
            using the {\PROD} Prolog documentation and
            delivery  system.
}

\end{verbatim}}
\caption[A sample {\PROD} preamble.]{A sample {\PROD} preamble from {\tt prod1.pl}. The results
of this preamble can be viewed at
\protect\url{http://tim.menzies.com/pdf/prod1.pdf}.}\label{fig:samplepreamble}
\end{figure}

\begin{figure}
{\scriptsize \begin{verbatim}
@Book{bratko01,
  Author =   "I. Bratko",
  Title =    "Prolog Programming for Artificial
                  Intelligence. (third edition)",
  Publisher =    "Addison-Wesley",
  Year =     2001
}

@article{me89zb,
  author =   "T.J. Menzies",
  title =    "Domain-Specific Knowledge Representations",
  month =    "Summer",
  journal =  "AI Expert",
  year =     "1989",
}

@InProceedings{menz91,
  AUTHOR =   "T.J. Menzies",
  YEAR =     " 1991",
  TITLE =    "{ISA} {O}bject {PARTOF}
              {K}nowledge {R}epresentation (Part Two)?",
  BOOKTITLE =    " Tools Pacific 4",
  EDITOR =   " B. Meyer",
  Note =     "Available from
     \url{http://tim.menzies.com/pdf/tools91.pdf}"
}

@PhdThesis{me95,
  AUTHOR =   " T.J. Menzies",
  YEAR =     " 1995",
  TITLE =    " Principles for Generalised
                Testing of Knowledge  Bases",
  School =   " University of New South Wales",
  Note =     "Ph.D. thesis. Available from
     \url{http://tim.menzies.com/pdf/95thesis.pdf}"
}

@TechReport{me96c,
  Author =   "T. Menzies and P. Haynes",
  Title =    "Empirical Observations of Class-level
               Encapsulation and Inheritance",
  Institution =  "Department of Software Development,
                  Monash University",
  Year =     1996,
  Note =     "Available from
        \url{http://tim.menzies.com/pdf/96encap.pdf}"
}

@InCollection{mich90,
  author =   {R.S. Michalski},
  editor =   {B.G. Buchanan and D.C. Wilkins},
  booktitle =    {Reading in Knowledge
                 Acquisition and Learning},
  title =    {Toward a Unified Theory of Learning},
  publisher =    {Morgan Kaufmann},
  year =     1993,
  pages =    {7-38}
}

@unpublished{spinmanual,
  author = "{G}erard {J}. {H}olzmann",
  title = "{B}asic {SPIN} {M}anual",
  note = "{A}vailable at
  \url{http://cm.bell-labs.com/cm/cs/what/spin/Man/Manual.htm}
"}

@Manual{swiprolog,
  Title =    "SWI-Prolog",
  Author =   "Jan Wielemaker",
  Note =     "Available from
    \url{http://swi.psy.uva.nl/projects/xpce/SWI-Prolog.html}."
}

\end{verbatim}}
\caption{A sample {\LaTeX} citation database.}\label{fig:bib}
\end{figure}


\subsubsection{The Body}

The {\em body} of a {\PROD} file contains Prolog source code
embedded in {\LaTeX} commands. Within the body, the
following conventions
hold:

\bi  \item Anything found between {\tt \%<} and {\tt \%>} is
preserved as verbatim text (e.g. see lines 44 to 48 of
\fig{prod0sample}).  \item A line starting with {\tt
\%{$\backslash$}command} is converted to {\tt
{$\backslash$}command} (e.g. line 25 of \fig{prod0sample}). \item
A {\em level 1 heading} is declared for text found after {\tt
\%\%\%\%} (e.g. line 37 of \fig{prod0sample}).
 \item A {\em level 2 heading} is declared for text
found after {\tt \%\%\%} (e.g. line 31 of \fig{prod0sample}).
 \item A {\em level 3 heading} is declared for text
found after {\tt \%\%} (e.g. line 34 of \fig{prod0sample}).
 \ei

In the case of level 1,2,3  headings:
\bi \item There can be no characters to the left of the comment
characters. \item If the line ends in {\tt \%<}, then the heading
is declared and verbatim text starts straight after the heading.
\ei

Currently, {\PROD} does not support headings levels greater than
3.
\clearpage
