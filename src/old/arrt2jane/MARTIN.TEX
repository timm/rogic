\documentclass[10pt]{article}
\usepackage{times,alltt,url,graphicx,amssymb}

\newcommand{\fig}[1]{Figure~\ref{fig:#1}}

\title{MARTIN v0.1: \underline{M}enzies' version of the
\underline{A}r\underline{RT} \underline{IN}terpreter}
\author{Tim Menzies$^1$\\{ \url{tim@menzies.com}}}
\begin{document}
\maketitle


 \section{Hello} MARTIN is $\approx$ 250 lines of Prolog
that is a ghost of the ARRT system~\cite{fea01}.


eg002.pl (see \fig{eg}) is stuff in the format of what you are generating from
arrt
\begin{figure}
 {\small
\begin{alltt}:- [arrt].
+source(v1,u0,when(2001,4,6,17,45,32)).
+aka(goal,"goal",v1).
+aka(u0,"Baseline",v1).
+aka(r9,"Top Level Requirement 1",v1).
+aka(r11,"Subsidiary Requirment 1.1",v1).
+aka(r12,"Subsisiary Requirement 2.1",v1).
+aka(r10,"Top Level Requirement 2",v1).
+aka(f13,"Failure Mode 1",v1).
+aka(f14,"Failure Mode 2",v1).
+aka(f15,"Failure Mode 3",v1).
+aka(p16,"PACT 1",v1).
+aka(p17,"PACT 2",v1).
+aka(p18,"PACT 3",v1).
+r(r10,10,1,0,v1).
+r(r12,1,1,0,v1).
+r(r11,1,1,0,v1).
+r(goal,1,0,0,v1) :- r9.
+r(r9,  1,0,0,v1) :- r11,r12,r10.
+f(f15,0,1,0,v1).
+f(f14,0,1,0,v1).
+f(f13,0,1,0,v1).
+p(p18,0,0.5,100,v1).
+p(p17,0,0.5,10,v1).
+p(p16,0,0.5,1,v1).
+impact(f13,r11,0.1,v1).
+impact(f14,r11,0.2,v1).
+impact(f15,r11,0.3,v1).
+impact(f13,r12,0.4,v1).
+impact(f14,r12,0.5,v1).
+impact(f15,r12,0.6,v1).
+effect(f14,p16,0.22,v1).
+effect(f15,p16,0.33,v1).
+effect(f14,p17,0.55,v1).
+effect(f15,p17,0.66,v1).
+effect(f14,p18,0.88,v1).
+effect(f15,p18,0.99,v1).
\end{alltt}}
\caption{A sample KB exported from ARRT.}\label{fig:eg}
\end{figure}

Internally, MARTIN's models are a directed graphs of nodes and
edges. This graph is shown  textually in \fig{text} and graphically in \fig{one}.

\begin{figure}
\begin{center}
\begin{minipage}{.45\linewidth}
{\small
\begin{alltt}
edge(r9, goal, 1, +).
edge(r11, r9, 1, +).
edge(r12, r9, 1, +).
edge(r10, r9, 1, +).
edge(f13, r11, 0.9, *).
edge(f14, r11, 0.8, *).
edge(f15, r11, 0.7, *).
edge(f13, r12, 0.6, *).
edge(f14, r12, 0.5, *).
edge(f15, r12, 0.4, *).
edge(p16, f14, 0.22, -).
edge(p16, f15, 0.33, -).
edge(p17, f14, 0.55, -).
edge(p17, f15, 0.66, -).
edge(p18, f14, 0.88, -).
edge(p18, f15, 0.99, -).
\end{alltt}}
\end{minipage}\hfill\begin{minipage}{.45\linewidth}
{\small
\begin{alltt}
node(r, r10, 10).
node(r, r12, 1).
node(r, r11, 1).
node(r, goal, 0).
node(r, r9, 0).
node(f, f15, 0).
node(f, f14, 0).
node(f, f13, 0).
node(p, p18, 0).
node(p, p17, 0).
node(p, p16, 0).
\end{alltt}}
\end{minipage}
\end{center}
\caption{A text dump of the internals of MARTIN.}\label{fig:text}
\end{figure}

\begin{figure}
\begin{center}
\includegraphics[width=5.5in]{report.eps}
\end{center}
\caption{A graphical view of \fig{text}.}\label{fig:one}
\end{figure}

At runtime, some faults and pacts are declared active and the coverage of the top most {\tt goal} is computed.
At the end of this paper (in \fig{puts})
are
 10 runs (each assumes some randomly selected subset of the
 faults and pacts are active).
\begin{figure}
\begin{center}
\begin{minipage}{.33\linewidth}{\small
\begin{alltt}
IF  true
    f13
    f14
    f15
    p17
    p18
THEN    f13 = 1
    f14 = 0
    f15 = 0
    r10 = 10
    r11 = 0.1
    r12 = 0.4
    goal = 10.5
    r9 = 10.5
--------------------
IF  true
    f13
    p16
    p17
    p18
THEN    f13 = 1
    r10 = 10
    r11 = 0.1
    r12 = 0.4
    goal = 10.5
    r9 = 10.5
--------------------
IF  true
    f13
    f14
    f15
    p16
    p18
THEN    f13 = 1
    f14 = 0
    f15 = 0
    r10 = 10
    r11 = 0.1
    r12 = 0.4
    goal = 10.5
    r9 = 10.5
\end{alltt}}\end{minipage}\hfill\begin{minipage}{.33\linewidth}
{\small \begin{alltt}
IF  true
    p16
    p17
THEN    r10 = 10
    r11 = 1
    r12 = 1
    goal = 12
    r9 = 12
--------------------
IF  true
    f13
    f14
    f15
    p16
    p17
    p18
THEN    f13 = 1
    f14 = 0
    f15 = 0
    r10 = 10
    r11 = 0.1
    r12 = 0.4
    goal = 10.5
    r9 = 10.5
--------------------
IF  true
    f13
    f14
    f15
    p17
THEN    f13 = 1
    f14 = 0.45
    f15 = 0.34
    r10 = 10
    r11 = 0
    r12 = 0.039
    goal = 10.039
    r9 = 10.039
\end{alltt}}\end{minipage}\hfill\begin{minipage}{.33\linewidth}
{\small \begin{alltt}
IF  true
    f13
    f14
    f15
    p18
THEN    f13 = 1
    f14 = 0.12
    f15 = 0.01
    r10 = 10
    r11 = 0
    r12 = 0.336
    goal = 10.336
    r9 = 10.336
--------------------
IF  true
    f13
    f14
THEN    f13 = 1
    f14 = 1
    r10 = 10
    r11 = 0
    r12 = 0
    goal = 10
    r9 = 10
--------------------
IF  true
    p16
THEN    r10 = 10
    r11 = 1
    r12 = 1
    goal = 12
    r9 = 12
--------------------
IF  true
    f13
    p16
    p17
    p18
THEN    f13 = 1
    r10 = 10
    r11 = 0.1
    r12 = 0.4
    goal = 10.5
    r9 = 10.5
\end{alltt}}
\end{minipage}
\end{center}
\caption{10 random outputs}\label{fig:puts}
\end{figure}


\bibliographystyle{plain}
\bibliography{/tim/wp/refs}

\end{document}
