\section{Style notes}

\subsection{{\LaTeX} Requirements}

Predicates can't have long argument lists. Practically speaking,
this often implies that you need DCGs to handle carrying round
updatable state.

Divide code up into lots of little files. If it can't fit into one
column, it is too big. In practice, this means  no file is more than
60 characters wide and 80 lines long.

Source code should be included into the {\LaTeX} source using the
`\code{file}{Caption}` command. This will generate a {\LaTeX}
reference `fig:file` which can be accessed in the usual way using
`\ref{fig:file}`.

For application `X` create one main file called `X.pl` (e.g., see
\fig{starlog.pl}). Include this main file into your {\LaTeX} source.

Anytime a Prolog program loads a file, it should be followed by a
{\LaTeX} figure reference. For example:
\newcommand{\bslash}{$\backslash$}

\begin{LISTING}
:- [abc]. % see {\bslash}fig\{abc.pl\}
\end{LISTING}

This way, if you forget to include `abc.pl` , you'll get one of those
``?'' warnings that {\LaTeX} generates (so you'll know if you've
missed anything).

You application will comprise: \bi \item {\em One main file} that
loads everything else; \item {\em Many library files} that set
things up; \item {\em Some core files} that combine the libraries
to perform
 some interesting task;
 \item
 {\em Some  example files} that
show off the core code. \ei For every core file `core.pl`, there
should be a `coreeg.pl` example file containing   a `coreeg`
predicate that does the showing off. `coreeg.pl` should have  the
following structure:

\begin{LISTING}
:- [lib1 % see {\bslash}fig\{lib1.pl\}
   ,lib2 % see {\bslash}fig\{lib2.pl\}
   ].
...
ccode :- doSomething.
:- demos(coreeg). % call to the core code that traps output
\end{LISTING}

A file with this structure will generate a file `coreeg.out` as a
side effect of loading `coreeg.pl`. In order to include the core,
example, and output files, the macro `\showtell{core}` is useful. The
call `\showtell{core}` generates three figures: one for `coreeg.pl`,
one for `coreeg.out` and one for `core.pl`.

It is good practice to reference the core, example, and output
files in the main file; e.g.
\begin{LISTING}
:-  [abc]. % see {\bslash}fig\{abc.pl\},  {\bslash}fig\{abceg.pl\} &  {\bslash}fig\{abceg.out\}
\end{LISTING}



Never use a backslash such as `\X` so: \bi \item Use `not(X)` instead
of `\+ X` \item Use `nl` instead of  `\n` . \ei

Never use `{X}` since {\LaTeX} can't print that. So: \bi \item Define
`goal_expansion/2` to repair over-zealous DCG expansions (see
\fig{hacks.pl}). \item Use `code(X)` to denote code that should be
not be DCG-expanded. Then add the following item to `hooks.pl`

\begin{LISTING}
goal_expansion(call(A,B,B),A).
\end{LISTING}

\ei
\subsection{Other Requirements}

\subsubsection{Peekers}

A {\em peeker} is a set of predicate sthat  ``looks before it
leaps''. The idea is that before doing anything on `X`, the code
peeks at `X` and then uses what it sees to determine what actions to
perform. Peekers, as used in this code, are filters that convert `X`
to `Y`. Their general form is three predicates: `p,p0,p1`:

\begin{LISTING}
p(X,Y) :- once(p0(X,Z)), p1(Z,Y).

p0(X,          barph(X) ) :- var(X). % usual top peeker
p0(X,  barph(number(X)) ) :- number(X). % e.g.
p0((X,Y),         (X,Y) ). ...
p0(X,       standard(X) ). % some final default action

p1(barph(X),  _)   :- print(bad(X)), fail.
p1(standard(X),Y)  :- Y is X % e.g.
p1((X0,Y0), (X,Y)) :- p(X0,X), p(Y0,Y). ...
\end{LISTING}

The `p0` predicate classifies the incoming structure into one of $N$
terms. One `p1` predicate is written for each such term. If `p1`
recurses, then the recursive call is back to `p`.

Peekers have several  useful features: \bi \item Fewer/no cuts in the
code. \item Stepping through the `spy` outputs over for a peeker is
usually easier since only one choice for `p1` is ever shown and the
`p0` processing is often simple enough that the programmer can just
skip it. \item Often there are less `p1` predicates than `p0`
predicates since `p0` might map different incoming terms to the same
functor (e.g., in the above, while there are two situations where we
want to `barph`, we only needed to write `barph` once. \item If `p0`
is constrained to key just on features available when the source code
is loaded for the first time, then the peeker can be optimized as
follows. At load time, pre-compute and cache all the `p0` results on
the loaded code. Then, at runtime, replace all the `p0` with
`p0(X,X)` and all that lookup computation disappears. \ei
