\documentclass[twocolumn,global]{sys/svjour}

\journalname{Submitted to WVU knowledge engineering} % banner on page 1
\date{WP ref: tim/src/pl/nb/nb.tex \today} % wp info
\usepackage{times} % best for viewing widely on acrobat
%\topmargin -1.5cm % move page up and down

\newcommand{\ME}{NB} % place name of software here

\usepackage{sys/timlog} %
\usepackage{sys/timtex} %

\begin{document}
\title{Some Prolog Macros for Rule-Based Programming: Why? How?}
\author{Tim Menzies\inst{1}, Santa Clause\inst{2}}

\institute{ Lane Department of Computer Science,
       University of West Virginia,
       PO Box 6109, Morgantown,
       WV, 26506-6109, USA;\\ \url{http://tim.menzies.com}; \email{
       tim@menzies.com} \and
 Child Gift International,
\email{santa@northpole.earth} } \maketitle
%\copyrightspace
% If you want to print drafts of the paper with a draft
% notice in the copyright space, comment out the \copyrightspace
% line above and include the \submitspace line below instead.
%
%\submitspace{Draft of paper submitted to ASE 2000.
%\\WP:b/00/ase/whatif(\today).}

\thispagestyle{empty}  % suppresses page number on first page
\pagestyle{plain} % places numbers on all pages
% pagestyle{empty} % removes numbers from all pages
%\setcounter{tocdepth}{4} \tableeofcontents % TOC control
% \listoffigures % LOF control

\begin{abstract}
The history, benefits, and drawbacks to pure rule-based programming
is discussed. A simple extension to pure rule-based programming is
described. The extensions are very quick to code and can be   easily
customized to support a range of knowledge engineering applications.
\end{abstract}

\section{Introduction}
At a workshop on rule-based programming (hereafter, RBP), it may be
heresy to say that there is more to knowledge than just rules.
However, after many years of commercial and research work on RBP, we
assert that this is so.

This article reviews some of the history of RBP and the need to
extend certain aspects of RBP. These extensions are simple to
implement- so simple in fact that the entire source code for those
extensions can be presented in this article.

\section{A Dummies Guide to RBP}

\subsection{Origins \& Early Successes}

This article focuses on rule-based knowledge engineering. Hence, by
``RBP'', we really mean ``how rules were used in classical knowledge
engineering''.

Much of the early 1980s hype surrounding commercial applications of
artificial intelligence came from early successes  with
 rule-based {\em production
systems}. Such systems were rule-based systems that queried and
updated objects in a {\em working memory} using a MATCH-SELECT-ACT
cycle: \bi \item MATCH: find the rules with conditions satisfied by
the current contents of working memory; \item SELECT: pick one rule
from the MATCHed set using a {\em conflict resolution strategy};
\item ACT: perform the action of the picked rule.\ei There are many
advantages to pure RBP. For example,  the uniformity of the RPG
paradigm makes it amenable to: \bi \item formal analysis of their
reliability, e.g.~\cite{chen95}; \item powerful learning schemes and
languages, e.g.~\cite{laird86}; \item the rapid creation of
high-productivity programming environments,
e.g.~\cite{lukose99,vandebrug86,marcus89}; \item the rapid training
of business users so that they  can create their own rule bases,
e.g.~\cite{mebfd92,men87a}; \item powerful
 maintenance environments, e.g.~\cite{preston93,compton92}. \ei

Further, RBP is  an insightful theoretical tool for cognitive
psychology.  Pure RBP can replicate certain expert and erroneous
behavior of experts. For example, one way to explain the difference
between expert and novice performance is that novices fill their
working (human) memory with an excess of active goals. This leaves no
room for any intermediaries of any particular calculation. On the
other hand, experts have  compiled their experience into
high-priority rules that select the next best action.  Hence, the
working memory of an expert has less active goals which means
experts are free to use their memory to run
computations~\cite{simon80}.

Not only is RBP useful for cognitive theory, it is a useful tool for
pragmatic software engineers.
 RBP enables a novel  iterative  and exploratory software  development methodology.
Iterative and exploratory software development is very useful when
prototyping software. Such prototyping is not required for
well-defined tasks. Such well-defined tasks can be implemented via a
  ``waterfall'' development process; i.e.

 \[waterfall\;=\;analysis \rightarrow
design \rightarrow code \rightarrow test\]

%XXX JESS
For less-defined tasks, waterfall development can stagnate in the
analysis stage since not enough is known about the domain. An
alternative approach is to use
 RBP to
generate an executable version of the current conceptualization of a
system.  Since each rule is a separate chunk of knowledge, it is easy
to quickly add more rules. On execution, the interaction of these
rules can lead to surprising results that prompt clarifications and
extensions of domain knowledge. This approach has been called various
things including ``knowledge elicitation via irritation'' or the RUDE
model; i.e.

\[RUDE\;=\;\underline{R}un \rightarrow \underline{U}nderstand
\rightarrow \underline{D}ebug \rightarrow \underline{E}dit\]



RBP methods resulted in the ``AI spring'' of the 1980s. Many
well-documented, mature, and optimized RBP systems were developed
such as ART \footnote{From Inference Corporation}, CLIPS\footnote{The
``C'' Language Integrated Production System, developed by
NASA~\cite{nasa91}}, and OPS5~\cite{brownston92} (just to name a
few). Numerous significant  RBP systems were developed including the
commercially successful XCON computer configuration
system~\cite{mcderm93}.


\subsection{Problems with Rules}



 The blossoming of RBP in the AI spring was not followed by an
RBP  summer. An assumption underlying the RUDE approach was  the {\em
RUDE assumption}; i.e:

\begin{quote}
{\em Rules are independent chunks of knowledge which can be easily
added or changed or removed.} \end{quote}

This proved not to be the case.  For example, once XCON grew to
10,000 rules, the developers of XCON had a RUDE\footnote{Pun.
Function: noun. Etymology: perhaps from Italian "puntiglio" which
means fine point or quibble. Definition: the usually humorous use of
a word in such a way as to suggest two or more of its meanings or the
meaning of another word similar in sound.} awakening: maintaining
XCON's rules had become fiendishly complicated. To some extent, this
was due to the density of knowledge within XCON: \bi \item The
expertise within XCON's rules reflected DEC's state-of-the-art
knowledge in configuring computers. \item Such a rich library of
knowledge will be intricate to maintain, no matter how it is
expressed. \ei However, another factor that complicated XCON's
maintenance was that its rules violated the RUDE assumption.
Real-world rule bases often contained groups of rules with
significant interactions. For example: \bi \item A careful reverse
engineering of XCON showed that the system executed though several
{\em operator spaces} where methods for improving the design of a
computer were carefully collected, rejected, elaborated, or assessed,
before the appropriate best {\em operator} was finally
selected~\cite{bachant84}. \item \fig{smallitems.pl} shows three
rules that check for certain special cases of bagging groceries.
These rules are not independent. Rule {\em b11} tries to sneak small
items into grocery bags that aren't full and which don't contain
bottles. If {\em b11} fails, then rule {\em b12} just places small
items into any grocery bag at all. Rule {\em b13} creates a new bag
when neither of the other two rules can find a bag for small items.
Note the tacit reliance of {\em b12} on {\em b11} handling a certain
special case (bags with bottles). Note also the tacit reliance of
{\em b13} on the other two rules: creating empty grocery bags is a
nonsense action {\em unless} some other agent tries to {\em first}
fill those bags. \ei


\code{smallitems.pl}{Three rules in the PIKE language (a STARLOG
variant). These rules access the object defined in
\protect\fig{grocery.pl}, \protect\fig{order.pl}, and
\protect\fig{bag.pl}. Example adapted from Winston's BAGGER
application~\cite{winston84}.}

 \code{grocery.pl}{A PIKE definition of  the GROCERY
object.}

\code{order.pl}{A PIKE definition of  the ORDER object.}
\code{bag.pl}{A PIKE definition of  the BAG object.}

The use of such coordinating rules violates the RUDE assumption since
every addition to the rule base has to be assessed with respect to
its effect on the rest of the rules.

Another problem with pure RBP is that the paradigm can confuse, not
clarify, certain types of procedural knowledge. Consider for example,
the process of finding the total volume of items in a grocery bag.
One {\em generator rule} is required for transferring pairs of
grocery items from that set to a temporary space of ``candidate
sums''. Another {\em intermediary rule} matches and deletes each
pair, then asserts the sum of their sum as another member of the
``candidate sums''. A final {\em report rule} waits till the
generator and intermediary rule stop firing, then accesses the
surviving ``candidate sum'' as the total volume of the grocery bag. A
more succinct representation of this procedural summation knowledge
that does not use rules is shown in the list summation procedure in
\fig{bag.pl} between \Line{bag.pl:volstart} and \Line{bag.pl:volend}.


Many other researchers argued that rules were not the appropriate
primitive construct of knowledge engineering. Despite careful
attempts to generalize the early RBP knowledge engineering work
(e.g.~\cite{stefik82}), the construction of knowledge-based systems
remained a somewhat hit-and-miss process. By the end of the 1980s, it
was recognized that  design concepts such as RBP were
incomplete~\cite{bucsmi89}. For example, Bobrow's reverse engineering
of real-world knowledge-based systems~\cite{bobrow85} found that
numerous paradigms were being employed including rule-based,
logic-based, functional, object-oriented, and ``access-based''
(which, these days, we might call implicit invocation \cite{shaw96}).
The 1990s was characterized by an extensive search for {\em
higher-level reusable patterns of inference } such as: \bi \item
Propose-and-revise (e.g. as done by~\cite{sch94}). \item Recursive
descent of ``problem spaces'' (e.g. as done \newline
by~\cite{yost93}). \ei

\subsection{Beyond RBP}

The above problems, and our own commercial knowledge engineering
(e.g.~\cite{men87a,mebfd92}), lead us to extent RBP. As done in many
other shells (e.g. ART, CLIPS), the need to use both procedural and
declarative rule knowledge made us combine RBP with a simple
object-oriented approach. Rule conditions and actions could use verbs
defined in the object-language. For example, rule {\em b12} on
\Line{smallitems.pl:notfull} of \fig{smallitems.pl} checks that a bag
is {\em notFull}, where {\em notFull} is a procedure defined at the
end of {\em bag} on \Line{bag.pl:notfull} of \fig{bag.pl}.

Also, for a while, we tried coding up knowledge engineering languages
based on the supposedly reusable {\em higher-level reusable patterns
of inference}. However, there was a problem. Our repeated experience
was that while small communities of experts might reuse an inference
pattern, that pattern was not widely endorsed elsewhere. That is,
while designing a rule-base around a certain inference pattern was
useful, each new application needed a new inference pattern (an
effect reported elsewhere~\cite{linster92}). More generally, while
many higher-level inference patterns have been identified (e.g.
propose-and-revise, heuristic classification, recursive descent of
``problem spaces'), the reusability of these patterns is questionable
since there never was widespread and stable agreement of the internal
structure of these patterns~\cite{me97j,me00w}.


\begin{figure}
\begin{center}
\includegraphics[width=3in]{iceberg.eps}
\end{center}
\caption{The ``iceberg model'' of knowledge
engineering.}\label{fig:iceberg}
\end{figure}

Even though inference patterns may not be reusable between domains,
they may be useful within a particular domain. Our default
architecture for a new knowledge based system was the {\em iceberg
model} of \fig{iceberg}. In that architecture, knowledge engineers
work ``under the waterline'' to build infrastructure to support the
``in view'' knowledge bases created by advanced business users. Our
role as knowledge engineers was to: \bi \item Identify cliches in the
expert's approach to different problems. Such cliches may include the
supposedly reusable inference pattern. \item Craft support code for
each such cliches. \ei Where possible, the support code was heavily
parameterized so that it could be extensively customized. These
customization parameters then became the ``tip of the iceberg'' that
was visible to our business users. These users then used these
upper-most drivers in their rules and objects. Our cliches included
low-level idioms such as summing all items in a list as well as
domain-specific high-level inference patterns.


\begin{figure}
\begin{center}
\includegraphics[width=3in]{growth.eps}
\end{center}
\caption{Patterns of rule growth. KE= knowledge
engineers}\label{fig:growth}
\end{figure}

\fig{growth} shows a typical pattern of authoring rules using this
iceberg approach. Note that the knowledge engineers write some rules
in the initial stages while, by the end of the development, the users
have written most of the rules. This pattern of rule authoring arises
from the following development methodology: \bd \item[{\em Language
development:}] Initially, the knowledge engineers struggle to
understand the domain and identify the relevant cliches.  After a
week or two, some of these cliches are found and implemented as
support code. \item[{\em Transition:}] The knowledge engineer then
builds a few sample rules to demonstrate the usage of the support
code. These sample rules are then used to train the business users.
\item[{\em Development:}] Business users go on to write most of the
knowledge base.\item[{\em Language elaboration:}] The knowledge
engineer watches their progress to identify common inference cliches
that are awkward or clumsy or error prone to encode. The knowledge
engineers then (i)~augment the support code and (ii)~show the
business users how to simplify their rules using the augmented
support code. As a result, the business users learn how to encode and
update their own rule base using a knowledge language that has been
heavily customized to their domain. \item[{\em Maintenance:}]
Maintenance in this approach is relatively  simple since
 business users can
update their own knowledge even when the knowledge engineer is
unavailable. \ed

\section{STARLOG}

The iceberg model is only possible when the practitioner can quickly
craft a new set of inference cliches. The rest of this paper
discusses STARLOG, a customization tool kit for knowledge
engineering.

The STARLOG system is a
 set of {\em load-time macros} that convert sentences
in some domain-specific terminology into a simple clause-based logic.
Since these macros are called at load time  then, in many cases, the
overheads of interpretation is incurred once at load time and never
at runtime.

 \code{ops.pl}{}

STARLOG is a Prolog-based framework for building different languages
for knowledge engineering. Prolog was chosen as the underlying
implementation language due to its ease of customization. For
example, \fig{ops.pl} changes the Prolog interpreter such that
certain user-friendly structures can be created in a pseudo-English
style; e.g. the rules shown in \fig{smallitems.pl}.

\code{aboutme.pl}

 \code{starlog.pl}{The idiom {\tt @[File1, File2,..]} is shorthand for
 ``don't load these
 files more that once unless they have not changed on disc and, if loading,
 don't print  verbose load messages''.}

This paper presents 367 lines of Prolog that implements a simple, but
useful rule/object interpreter/optimizer. The main load file of
STARLOG is shown in \fig{starlog.pl}. Of this code, 127 lines are
support utilities; 153 implement an optimized simple object language
and 87 lines implement a forward chaining rule-based system. Such a
small set of utilities can easily be customized for new domains. One
such customization is the PIKE language\footnote{For Star Trek
aficionados, we offer the following notes. STARLOG variants should be
named, in order, after the  captains of the Rodenberry-class star
ships: i.e.  PIKE KIRK, SPOCK, PICARD, SISKO, JANEWAY, DA'AN and
ARCHER.
 The names
 STOCKER, DECKER, KAHN and ZO'OR are reserved
 for throw-away crazy prototypes,
 for obvious reasons.} used for the rules and objects shown in
\fig{smallitems.pl}, \fig{grocery.pl}, \fig{order.pl} and
\fig{bag.pl}.

\code{hooks.pl}{}

PIKE supports three main constructs shown in \fig{hooks.pl}: objects,
methods, and rules. These constructs are discussed below, after
introducing a sample PIKE application.


\section{A Sample Application}

This paper contains full source code for a PIKE rule/object system
that extends Patrick Winston's BAGGER problem~\cite{winston84}.
BAGGER is Winston's allegory for XCON: XCON configures computers by
checking the right components are combined together while BAGGER
checks that the right grocery orders are combined together in grocery
bags. Our extension includes rules, objects and methods for
groceries, bags, and orders.

\code{ruleseg.pl}{}

 PIKE's BAGGER is loaded in \fig{ruleseg.pl}.
The system contains five rule groups: \bd \item[{\tt Global:}] This
is the initial group. It creates a sample order; see
\fig{ruleseg.pl}. The current group is then changed to...
 \item[{\tt Check\_order:}]
 Checks if any items are missing from the order; see
 \fig{checkrules.pl}.
 The current group is then changed to...



\item[{\tt Bag\_large\_items:}]
 Handles the bagging of the bulky items; see
 \fig{largeitems.pl}.
 The current group is then changed to...



\item[{\tt Bag\_medium\_items:}]
 Handles the bagging of the mid-sized items; see
 \fig{mediumitems.pl}.
 The current group is then changed to...

\item[{\tt Bag\_small\_items:}]
 Tries to sneak the small items into the bags created above; see
 \fig{smallitems.pl}.


\ed



\code{checkrules.pl}{} \code{mediumitems.pl}{}



\fig{ruleseg.out} shows what happens when the whole system is loaded
and executed. Given the ORDER {\small
\begin{alltt}
[bread, glop, pizza, granola,
 iceCream, potatoChips]
\end{alltt}
} PIKE's BAGGER generates two bags:

{\small
\begin{alltt}
bagDB(1, [glop, potatoChips, iceCream, bread]). bagDB(0, [granola,
pizza, pepsi]).
\end{alltt}
}

\code{ruleseg.out}{}
%\code{ruleseg.out}{}


%\input{timtex}

%\input{todo}

\section{Inside STARLOG and PIKE}

This section discusses some of the inner details of our
implementation. Due to space restrictions, this discussion will be
quite terse. A longer version of this article is under preparation
which explains the system in more detail.

\subsection{Objects}

Knowledge base authors (hereafter, authors) can define objects using
the idiom \mbox{{\tt Helper=Spec}} where {\tt Spec} describes the
structure of a working memory object.
\code{speceg.out}{\outputfrom{spec}}
\code{speceg.pl}{\generates{spec}} Each PIKE object contains  a set
of named fields, some of which are marked by a {\tt +} if they are
index fields. GROCERY items have no indexes (see
\Line{grocery.pl:definition} of \fig{grocery.pl}) but ORDERs (see
\Line{order.pl:definition} of \fig{order.pl}) and BAGs (see
\Line{bag.pl:definition} of \fig{bag.pl}) have one indexed field
each. {\tt Helper} is the name of a predicate that can access parts
of an object. For example, \fig{speceg.out} shows the internal Prolog
representation of \fig{bag.pl}. The {\tt grocery/5} predicates
starting at \Line{speceg.out:g5} allow the access and update of
fields within a GROCERY object.


 The \mbox{{\tt Helper=Spec}} idiom is handled by the {\tt
 spec/2}
 predicate defined in \fig{spec.pl}. Currently, PIKE's objects support encapsulation
 and polymorphism, but not inheritance.

\code{spec.pl}{\corefile{spec}}


\subsection{Methods}

The idiom  \mbox{{\tt Helper*Head --> Body}} is PIKE's method syntax.
These methods are an {\em extension } of the standard Prolog definite
clause grammar and so are called   {\em ECG}s. ECGs can contain named
references to fields. Within ECGs, the idiom {\tt `X} accesses the
current value of field {\tt X} and the idiom {\tt
!X} updates field {\tt X}.  %XXX explain the wrapper
For example: \bi \item \LINE{grocery.pl:functor} of \fig{grocery.pl}
has the code \mbox{{\tt functor(`type,T,_)}}. STARLOG's load-time
macros expand this into the access of the third field of {\tt
groceryDB/5}  (i.e. the {\tt type} field) shown in
\LINE{speceg.out:access} of \fig{speceg.out}. \item
\LINE{bag.pl:update} of \fig{bag.pl} has the code {\tt !id=Id}. This
idiom forces an update of the first field of a {\tt bagDB/2} fact to
a value computed from the {\tt flag/3} predicate\footnote{SWI
Prolog's {\tt flag/3} predicate maintains and updates numeric
counters.}; i.e.: \ei

{\scriptsize
\begin{alltt}
bag(newBag(Id,Contents),_,bagDB(Id,Contents)):-
    flag(ids,Id,Id+1)
\end{alltt}
}

\noindent The  \mbox{{\tt Helper*Head --> Body}} idiom is handled by
{\tt ecg/2} defined in \fig{ecg.pl}.

\code{ecg.pl}

 \code{largeitems.pl}{}



\subsection{Forward Chaining Rules}

The idiom  {\tt Label if Condition then Action} is PIKE's way of
defining forward chaining rules. Rules have {\em priorities} and {\em
groups} which can be specified within the {\em Label} The default
group and priority is {\em global} and 10, respectively. For example,
\Where{largeitems.pl}{label} shows a rule being entered into the {\em
bag\_large\_items} rule group.

 \code{rules.pl}{}



Internally, the rule {\small
\begin{alltt}
Id in Group at Priority if Condition then Action
\end{alltt}
} is converted into two Prolog predicates

{\small
\begin{alltt}
lhs(Group,Priority,Id,Memory) :- Condition
\end{alltt}
} and {\small
\begin{alltt}
rhs1(Group,Id,Memory) :- Action
\end{alltt}
} (see \Where{rules.pl}{lhs} and \Where{rules.pl}{rhs}). This
separation permits the extensive customization of the forward chainer
since rule conditions can be tested without triggering the rule
action.

\code{fchain.pl}{}

The variables {\tt Group,Priority,Id,Memory} are used by PIKE's {\em
conflict resolution strategies}. Recall that conflict resolution is
the process of selecting one rule from the space of rules of
satisfied conditions. PIKE employs the following conflict resolution
strategies: \bd \item[{\em Rule groups:}] PIKE maintains a pointer to
the current group in the {\tt group/1} fact (see
\Where{fchain.pl}{group}). Only rules within the current group are
tested. \item[{\em Priority ordering:}] Prior to forward chaining,
PIKE gathers together a list of all the unique group names and rule
priorities within each group (see \Where{fchain.pl}{gather}). At
runtime, rules are explored within a group in priority ordering
starting with priority one and continuing to lower priorities (see
 \Where{fchain.pl}{todo1} and   \Where{fchain.pl}{todo2}).
\item[{\em Refraction:}] PIKE never fires the same rule action twice
on the same set of variable bindings. The {\tt Memory} argument of
{\tt lhs/4} and {\tt rhs1/3} contains all the variables passed from
the {\tt Condition} to the {\tt Action}. These shared variables are
found via {\tt sharedVars/3} shown in \fig{sharedvars.pl} which is
called at \Where{rules.pl}{shared}. \item[{\em Recency:}] When PIKE
asserts anything, it is asserted
 above all older assertions (e.g. see \Where{speceg.out}{sets} and \Where{speceg.out}{makes}).
 Hence, rules will fire more on newer
 assertions than older assertions. \ed

\code{sharedvars.pl}

\subsection{The {\tt With} Statement}

Another important idiom within PIKE is \[\mbox{{\tt Class=Id with
Method1 with Method2 with ...}}\] PIKE expands these to multiple
method calls invoked over the same object. Important variants of this
idiom are: \bi \item \mbox{{\tt change Class=Id with Method1 with
...}}\newline The {\em Methods} are prefixed  by a match to the
object   and are followed by an update to the object. \item
\mbox{{\tt make Class  with Method1 with ...}}\newline The {\tt
Methods} are run on a new object of the specified {\tt Class} and are
followed by an assertion of the resulting object. \item \mbox{{\tt
zap Class=Id  with Method1 with ...}}\newline The {\tt Methods} are
prefixed  by a match to the object   and are followed by deletion of
the object. The {\tt Method}s are called prior to deletion. \item
\mbox{{\tt zap Class=Id}}\newline The {\tt Methods} are prefixed  by
a match to the object   and are followed by deletion of the object.
\ei The prefix and following code is added between
\Where{ecg.pl}{wrap0} and \Where{ecg.pl}{wrap1}

\code{tidy.pl}{Remove redundant trues.}

\subsection{Optimizations}

PIKE also includes two optimization methods: unfolding and a
unification of  the match/select process: \bd \item[{\em Unfolding:}]
If a clause sub-goal matches to a single other clause, then the
sub-goal is replaced by the body of the other clause; see
\Where{spec.pl}{x1}. Sometimes, this unfolding unwraps to just a {\tt
true} sub-goal; see \Where{spec.pl}{x2}. Such {\tt true} sub-goals
are redundant and are purged via the code of \fig{tidy.pl}.
\item[{\em Unified match/select:}]  The order in which PIKE's rules
are to be tested can be determined via the rule priority number. If
rules are tested within each group in this order, then the {\em
first} rule with a satisfied condition would be the highest priority
satisfied rule. By exploring rules in this order, PIKE avoids a
computationally expensive MATCH process.; see
 \Where{fchain.pl}{todo1} and   \Where{fchain.pl}{todo2}.
\ed







\subsection{The Rest of STARLOG}

The rest of STARLOG is contained in the following files: \bi \item
\fig{flags.pl} is the first file loaded, it sets certain global
flags. \item \fig{wrapper.pl} shows a predicate used on
\Line{ecg.pl:wrapper} of \fig{ecg.pl}. \item \fig{verbs.pl} defines
some of the verbs used in rules. \item \fig{egs.pl} is a file which,
if loaded, will exercise most of STARLOG's PIKE. \item \fig{hacks.pl}
contains certain Prolog hacks such as repair of over-zealous DCG
expansion. \item \fig{singleton.pl} tests if only one clause matches
a sub-goal. \item \fig{show.pl} is a tool for printing lists of
clauses. \item \fig{demos.pl} is a tool for running a goal and
trapping its output to a file. \item \fig{misc.pl} contains
miscellaneous code.

\ei

\code{flags.pl}{} \code{wrapper.pl}{} \code{verbs.pl}{}
\code{egs.pl}{} \code{hacks.pl}

\code{singleton.pl}

\code{show.pl}{A simple pretty print.}



\code{demos.pl}{Run a goal, trap its output to file and, also, show
it on the screen.}

\code{misc.pl}



\section{Conclusion}

Pure rule-based programming had many proponents in the early days of
knowledge engineering. These proponents became fewer in number as
more and more developers were forced to extend pure rule-based
programming.

We have argued here that such extensions are necessary and simple.
The Prolog code shown here is very brief and implements a object/rule
interpreter/optimizer. Such a small system is easily customized and
supports our preferred toolkit for knowledge engineering: the
encoding of domain-specific knowledge cliches.

\bibliographystyle{abbrv}
{\footnotesize \bibliography{../../refs}}

\appendix

\section{License}\label{sec:license}
\code{sys/license.pl}

This software is distributed under the  GNU General Public License.
\fig{license.pl} shows routines to display that license.

\subsection{nowarranty.txt}\label{sec:nowar}
{\scriptsize \ME~ \input{sys/nowarranty.txt}

}

\subsection{warranty.txt}\label{sec:war}
{\scriptsize\input{sys/warranty.txt}

}

\subsection{conditions.txt}\label{sec:cond}
{\scriptsize\input{sys/conditions.txt}

}

\end{document}
