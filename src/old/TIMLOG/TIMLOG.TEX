\documentclass[twocolumn,global]{timlog/svjour}

\journalname{Submitted to WVU knowledge engineering} % banner on page 1
\date{WP ref: tim/src/pl/nb/nb.tex \today} % wp info
\usepackage{times} % best for viewing widely on acrobat
%\topmargin -1.5cm % move page up and down

\newcommand{\ME}{NB} % place name of software here

\usepackage{timlog/timlog} %
\usepackage{timlog/timtex} %

\begin{document}
\title{\TimLoG: Simple Prolog Documentation in \LaTeX }
\author{Tim Menzies\inst{1}, Clarke Clause\inst{2}}

\institute{Lane Department of Computer Science,
       University of West Virginia,
       PO Box 6109, Morgantown,
       WV, 26506-6109, USA;\\
       \url{http://tim.menzies.com};
       \email{tim@menzies.com} \and
 Prolog Programming International
%\email{clause@prolog.net}
} \maketitle
%\copyrightspace
% If you want to print drafts of the paper with a draft
% notice in the copyright space, comment out the \copyrightspace
% line above and include the \submitspace line below instead.
%
%\submitspace{Draft of paper submitted to ASE 2000.
%\\WP:b/00/ase/whatif(\today).}

\thispagestyle{empty}  % suppresses page number on first page
\pagestyle{plain} % places numbers on all pages
% pagestyle{empty} % removes numbers from all pages
%\setcounter{tocdepth}{4} \tableeofcontents % TOC control
% \listoffigures % LOF control

\begin{abstract}
Simple macros written in \LaTeX allow for a simple documentation
scheme for Prolog.
\end{abstract}

\section{Introduction}
``Pretty good'' is good and ugly  is pretty bad. Standard
\LaTeX macros for pretty-printing Prolog source code (e.g. {\em\newline
lgrind.sty} focus on just making the source code look pretty. \TimLoG~
is a system for generating descriptions of Prolog source code;
i.e. not just source
code all the source code {\em and} explanatory text.

Services offered by \TimLoG are: \bi \item Via \LaTeX, the ability to
generate postscripts and acrobat files from Prolog source `code`.
\item Very simple inclusion of Prolog source code files. \item A
simple protocol for including outputs from Prolog programs into a
document. \item Line numbers on source code. \item The ability to add
``marks'' to source code and refer to these marks symbolically. \item
Cross references between source code. \item Auto-inclusion of the GNU
Public License into your system. \item Various useful little \LaTeX
tricks. \ei

Assuming  \TimLoG framework, then Prolog programmers can adopt
The above is implemented via the {\em least} number of extensions to standard
\LaTeX. The dream is that such a minimal implementation will be easy to extend and maintain.
We'll see!

\section{Installation}
Find an existing \TimLoG~ installation and copy into a new directory.
Also copy the  following files into the root of some new directory:
\bd
\item[{\tt timlog.tex}:] This file. Contains several examples of how to use
\TimLoG. Copy this file to (e.g.)  {\tt mine.tex} and then write your own
system by modifying this file.
\item[{\tt sys/*}:] Magic files you shouldn't have to modify.
\ed



\input{timlog/style}

\section{A Dummies Guide to RBP}

\subsection{Origins \& Early Successes}
 %\code{grocery.pl}{A PIKE definition of  the GROCERYobject.}

%\code{order.pl}{A PIKE definition of  the ORDER object.}

The use of such coordinating rules violates the RUDE assumption since
every addition to the rule base has to be assessed with respect to
its effect on the rest of the rules.


\section{Conclusion}

Pure rule-based programming had many proponents in the early days of
knowledge engineering. These proponents became fewer in number as
more and more developers were forced to extend pure rule-based
programming.

We have argued here that such extensions are necessary and simple.
The Prolog code shown here is very brief and implements a object/rule
interpreter/optimizer. Such a small system is easily customized and
supports our preferred toolkit for knowledge engineering: the
encoding of domain-specific knowledge cliches.

\bibliographystyle{abbrv}
{\footnotesize \bibliography{../../refs}}

\appendix

\section{License}\label{sec:license}
\code{license/license.pl}

This software is distributed under the  GNU General Public License.
\fig{license/license.pl} shows routines to display that license.

\subsection{nowarranty.txt}\label{sec:nowar}
{\scriptsize \ME~ \input{license/nowarranty.txt}

}

\subsection{warranty.txt}\label{sec:war}
{\scriptsize\input{license/warranty.txt}

}

\subsection{conditions.txt}\label{sec:cond}
{\scriptsize\input{license/conditions.txt}

}
\end{document}
