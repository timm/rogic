\documentclass[twocolumn,global]{tex4log/svjour}

\journalname{Submitted to WVU knowledge engineering} % banner on page 1
\date{WP ref: tim/src/pl/timlog/tex4log.tex \today} % wp info
\usepackage{times} % best for viewing widely on acrobat
%\topmargin -1.5cm % move page up and down

\newcommand{\ME}{NB} % place name of software here

\usepackage{tex4log/tex4log} %
\usepackage{tex4log/timtex} %

\begin{document}
\title{\Tex4Log: Simple Prolog Documentation in \LaTeX }
\author{Tim Menzies\inst{1}, Claude A. Clause\inst{2}}

\institute{Lane Department of Computer Science,
       University of West Virginia,
       PO Box 6109, Morgantown,
       WV, 26506-6109, USA;\\
       \url{http://menzies.us};
       \email{tim@menzies.us} \and
 Prolog Programming International
%\email{clause@prolog.net}
} \maketitle
%\copyrightspace
% If you want to print drafts of the paper with a draft
% notice in the copyright space, comment out the \copyrightspace
% line above and include the \submitspace line below instead.
%
%\submitspace{Draft of paper submitted to ASE 2000.
%\\WP:b/00/ase/whatif(\today).}

\thispagestyle{empty}  % suppresses page number on first page
\pagestyle{plain} % places numbers on all pages
% pagestyle{empty} % removes numbers from all pages
%\setcounter{tocdepth}{4} \tableeofcontents % TOC control
% \listoffigures % LOF control

\small

\begin{abstract}
Simple macros written in \LaTeX allow for a simple documentation
scheme for Prolog.
\end{abstract}


\begin{figure}
\begin{center}
\includegraphics[width=2in]{tex4log/ugly.eps}
\end{center}
\caption{~}\label{fig:ugly}
\end{figure}
\section{Introduction}

Gauguin once said that ``The ugly may be beautiful,  the pretty
never.''. Here, we strive to turn the ugly facts (and rules) of
Prolog into beautiful and attractive typeset prose.

We are not the first to attempt this goal (e.g., see \fig{ugly}).
 Standard \LaTeX macros for pretty-printing Prolog source
code (e.g. \url{lgrind.sty}) focus on just making the source code
look pretty. \Tex4Log~ is a system for generating descriptions of
Prolog source code; i.e. not just source code all the source code
{\em and} explanatory text.

Services offered by \Tex4Log are: \bi \item A framework for
organizing Prolog applications into lots of easily explainable units.
\item Via \LaTeX, the ability to generate postscripts and acrobat
files from Prolog source `code`. \item Very simple inclusion of
Prolog source code files. \item A simple protocol for including
outputs from Prolog programs into a document. \item Line numbers on
source code. \item The ability to add ``marks'' to source code and
refer to these marks symbolically. \item Cross references between
source code. \item Auto-inclusion of the GNU Public License into your
system. \item Various useful little \LaTeX tricks. \ei

The above is implemented via the {\em least} number of extensions to
standard \LaTeX. The dream is that such a minimal implementation will
be easy to extend and maintain. We'll see{\exclaim}

\section{Installation}
Find an existing \Tex4Log~ installation and copy the entire directory
tree into a new directory, say \url{dir}. Create a new sub-directory,
say \url{dir/myapp}. The main file of your new application will be
stored in \url{dir/myapp.pl}, will be documented in
\url{dir/myapp.tex} and the source code will be stored in in
\url{dir/myapp/}.



\input{tex4log/style}


\bibliographystyle{abbrv}
{\footnotesize \bibliography{../../refs}}

\appendix

\section{License}\label{sec:license}
\code{license/license.pl}

This software is distributed under the  GNU General Public License.
\fig{license/license.pl} shows routines to display that license.

\subsection{nowarranty.txt}\label{sec:nowar}
{\scriptsize \ME~ \input{license/nowarranty.txt}

}

\subsection{warranty.txt}\label{sec:war}
{\scriptsize\input{license/warranty.txt}

}

\subsection{conditions.txt}\label{sec:cond}
{\scriptsize\input{license/conditions.txt}

}
\end{document}
