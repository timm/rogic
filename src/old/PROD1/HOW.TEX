
\section{How?}
\subsection{Installing} {\PROD} comes as one flat directory with
lots of included files. Email me at \url{tim@menzies.us} for that
zip file. Just unzip it into a fresh directory.

If you just want to run  a {\PROD} application, all you need is a
Prolog interpreter. A {\PROD} file is a syntactically valid Prolog
program that can be loaded into a Prolog interpreter without
modification.

On the other hand, if you want to use {\PROD} to document your code,
they you'll need a working {\LaTeX}, Prolog and Perl installation.
Most UNIX installations have all three. But if you need to get your
own system going under Windows, then the software shown in
\fig{software} might be useful.

\begin{figure*}
{\footnotesize
\begin{center}
\begin{tabular}{|p{6in}|}\hline
 PERL:
        \bi
            \item[--$\checkmark$] Perl can be downloaded from many sources. For example, it comes as part
                of the {\em CYGWIN} distribution from \url{http://xfree86.cygwin.com}.
        \ei
\\\hline

PROLOG:
        \bi
            \item[--$\checkmark$]  Interpreters: SWI-Prolog \url{http://www.swi-prolog.org}.
            \item[--$\checkmark$]  Editors:
                \bi
                    \item[--] Some of my students speak highly of the {\em Prolog IDE} editor
                            \url{http://www.bildung.hessen.de/abereich/inform/skii/material/swing/indexe.htm}.
                    \item[--] I prefer {\em EMACS}, a Windows version of which can be found at
                            \url{http://www.gnu.org/software/emacs/windows}
                    \item[--$\$$] An excellent alternative to {\em EMACS} is {\em TEXTPAD}:
                                \url{http://www.textpad.com/download/}.
                            It has ignorable nag screens which can be removed for \$27 (ish).
                    \item[--] A simpler editor, that is free, and  has a smaller footprint, is {\em PFE}.
                          Its a very useful editor and it can be installed without super users.
                          \url{http://www.lancs.ac.uk/people/cpaap/pfe/}.
                \ei
        \ei
\\\hline
{\LaTeX}:
    \bi
        \item[--] Postscript processing
            \bi
                \item[--$\checkmark$]  {\em GHOSTSCRIPT} and {\em GSVIEW} are the core postscript processing utilities:
                        \url{http://www.cs.wisc.edu/~ghost}.
            \ei
        \item[--] A {\LaTeX} compiler:
            \bi
                \item[--$\checkmark$]  {\em MIKTEX} is a good Windows-based
                            {\LaTeX} distribution: \url{http://www.miktex.org}
                \item[--] {\LaTeX} training material can be found
                in many places including
                           \url{http://www.ling.upenn.edu/advice/latex.html}.
                           For this page you can find the very
                           excellent:
                                \bi
                                    \item Quick start directions:
                                    \url{http://www.ling.upenn.edu/advice/latex/starting.html}
                                    \item The Not So Short Introduction to
                                    {\LaTeX} (highly recommended):
                                    \url{ftp://ftp.tex.ac.uk/tex-archive/info/lshort/english/lshort.pdf}.
                                        This document may also be
                                        found with the standard
                                        {\PROD} distribution.
                                    \item Guide to Including Graphics
                                    \url{http://www.ling.upenn.edu/advice/latex/grfguide.pdf}
                                \ei


                \ei
        \item[--] Editing {\LaTeX}:
            \bi
                \item[--$\checkmark\$$] Under Windows,  {\em WINEDT}  is
                        the recommended  {\LaTeX} editor: \url{http://www.winedt.com}. It has some ignorable
                            nag screens which can be removed for \$30 (ish).
            \ei
         \item[--] Viewing the output. {\LaTeX} generates DVI files, postscript files, and Acrobat files.
            \bi
                \item[--$\checkmark$]  DVI files  can be viewed using the {\em YAP} viewer that comes with {\em MIKTEX}.
                \item[--$\checkmark$]  Postscript files can be viewed using the {\em GSVIEW} program from
                        \url{http://www.cs.wisc.edu/~ghost}.
                \item[--$\checkmark$]  The Acrobat reader can be downloaded from
                        \url{http://www.adobe.com/products/acrobat/readstep2.html}.
            \ei
        \item[--] Plotting scientific data:
            \bi
                \item[--] The {\em GNUplot} utility from  \url{http://www.gnuplot.vt.edu/}
                    can generate postscript plots of scientific data.
            \ei
        \item[--] Drawing packages:
            \bi
                \item[--$\$$] {\em MAYURA DRAW} is a  vector drawing utility for creating SVG and EPS illustrations:
                        \url{http://www.mayura.com/}. It can be used for free for 30 days (ish) then a registration
                            must be bought for \$30 (ish).
                \item[--] The amazingly useful, and very small, {\em jpeg2ps} converts any
                JPEG file to an eps:
                \url{http://www.pdflib.com/jpeg2ps/}. Now, any
                graphic that can be converted to a JPEG can be
                EPS-ed and included into a {\LaTeX} document.
                \item[--] And to convert anything to JPEG, use {\em IRFANVIEW}:
                    \url{http://www.irfanview.com}
                \item[--] Finally, if you can't import it any
                other way, get it on the screen, screen sieze it
                with {\em SCREENSIZE} (\url{http://www.pcmag.com/article2/0,4149,10206,00.asp}, copy and paste it to {\em
                IRFANVIEW} then {\em jpeg2ps} it.

            \ei
        \item[--] Auto-layout of directed and undirected graphs:
            \bi
                \item[--] {\em DOT}: The GRAPHVIZ distribution from Bell Labs
                        contains the {\em DOT} graph layout and
                        visualization tool:
                    \url{www.research.att.com/sw/tools/graphviz}.
                    {\em DOT} can generate EPS files.
            \ei
        \item[--] Spell checking {\LaTeX}:
            \bi
                \item[--]
                    The  {\em ISPELL} checker is a good UNIX-based spell checker.
                    Most UNIX installations integrate it with {\em EMACS}.
                \item[--]
                    {\em WINDEDT} has a good editor.
            \ei
       \ei

\\\hline
\end{tabular}
\end{center}}
\caption[Windows software for {\PROD}]{Support code for {\PROD},
under Windows. For a minimal installation, only get the items
marked with $\checkmark$. This software is freeware, except  the
items marked with $\$$.}\label{fig:software}
\end{figure*}



\subsection{How to load a {\PROD} system}

{\PROD} assumes that files come in a {\PROD}-compatible format.

\subsubsection{{\PROD}-compatible
applications}\label{sec:prodcompat}

A {\PROD}-compatible Prolog system comprises several files:

\be \item A main load file called, say, {\tt myfile.pl}. This main
load file loads up to three other files. \item {\tt myfile0.pl}: a
small set of pre-load actions. \item {\tt myfile1.pl}: the bulk of
the code. \item {\tt myfile2.pl}: start-up actions to be performed
after the the code is loaded. \item A documentation file called
{\tt myfile.pdf} auto-generated from {\tt myfile.pl}.\ee

\subsubsection{Sample pre-load actions in {\tt myfile0.pl}}

\bi \item{\em Loads of other Prolog systems}: In the case where you
are loading other {\PROD}-compatible  files, then  you'll have to
carefully inspect the pre-load and start-up actions of the systems
you are loading. In the best case, you can just load the main files
of the other {\PROD}s. However, sometimes you have to skip loading
those pre-load and start-up files, but weave their actions in with
your own pre-loads and start-ups. \item {\em Operator definitions}.
\item {\em Flags} such as what predicates are {\tt dynamic}. \item
{\em Hooks into the Prolog reader}: such as `goal_expansion/2` and
`term_expansion/2`. \item {\em Hacks}: those shameful things  we
can't avoid. So we   keep separate from the rest of our beautiful
code in a separate section. And we don't talk too much about them. So
lets go to a new section. \ei

\subsubsection{Start-up actions in {\tt myfile2.pl}}

These are application-specific and may include actions like loading
configuration files, then some domain-specific assertions, then
calling the main processing predicate of the system.


\subsection{How to document a {\PROD} system}

\subsubsection{Starting from scratch}

To start writing  {\PROD} code,   copy the {\tt template.pl}
(which comes with the standard {\PROD} distribution) and  rename
it to (e.g.) {\tt yourfile.pl}. Once that is done, then two
programs are required to convert  your code into a PDF format.

\[
yourfile.pl \overbrace{\longrightarrow}^{prep} yourfile.tex
\overbrace{\longrightarrow}^{{\mbox \LaTeX}} yourfilepdf
\]

The {\tt prep} and {\LaTeX} programs are described below.

\subsubsection{{\tt Prep}: converting {\tt *.pl} to {\tt *.tex}}

The pre-processor {\tt prep} converts the file (e.g.) {\tt
prep0.pl} to {\tt prep0.tex}. It is convenient to create a file
{\tt preps} that lists all your files that will need {\tt
prepping}. For example:

{\scriptsize\begin{verbatim}
perl prep file1
prep prep file2
\end{verbatim}}

When executed, this script looks for (e.g.) a {\tt file.pl} and
{\tt file2.pl} and generates  the files {\tt file1.tex}, {\tt
file2.pl}. Note that during that translation,  \bi \item The
characters {$\backslash$}{\tt *} and {\tt *}{$\backslash$} are
deleted. Hence, the characters on (e.g.) line 47 and 51 of
\fig{prod0sample} are deleted.   \item A line starting with {\tt
\%{$\backslash$}command} is converted to {\tt
{$\backslash$}command} (e.g. line 24 of \fig{prod0sample}). \ei

The resulting  {\tt *.tex} file can then be converted to PDF using
some {\LaTeX} system.

\subsubsection{{\LaTeX}: converting {\tt *.tex} to {\tt *.pdf}}

On my UNIX system, the script {\tt mytex}  generates PDF files
from {\LaTeX} files, then copies it to my web site. The command
line
\[
mytex\; prod0\; 03prod0
\]
takes \fig{prod0sample} and generates the file that can be viewed
at \url{http://tim.menzies.com/pdf/03prod0.pdf}.

Incidently, this file is
\url{http://tim.menzies.com/pdf/03prod.pdf} and was generated
using the command line
\[
mytex\; prod\; 03prod
\]

\begin{figure}
{\scriptsize \begin{verbatim}
 1 latex $1 > /dev/null
 2 grep "Warning:" $1.log
 3 bibtex $1 > /dev/null
 4 grep "Warning:" $1.blg
 5 latex $1 > /dev/null
 6 latex $1 >/dev/null
 7 dvips -q $1.dvi -o $1.ps
 8 ps2pdf $1.ps $1.pdf
 9 rm $1.ps               # save space- zap postscript file
10 cp $1.pdf $HOME/public_html/pdf/$2.pdf
11 chmod a+r
$HOME/public_html/pdf/$2.pdf
\end{verbatim}}
\caption[{\tt mytex}: generating pdf files from {\LaTeX}]{{\tt
mytex}: generating pdf files from {\LaTeX}, then copying the
result to a web-enable directory so it can be browsed. Assumes
that the directory {\tt \$HOME/public_html/pdf/} has already been
generated. The call to {\tt bibtex} on line 3 generates the
bibliography. The multiple passes through {\LaTeX} on lines 5 and
6 fix up all the bibliography and figure
references.}\label{fig:mytex}
\end{figure}

\subsubsection{Load order and documentation order}

Sometimes, the order in which you load files into Prolog is {\em not}
the order in which you want to explain an application. For example,
consider an application containing some low-level support code. The
support code may have to be loaded {\em first}, before the rest of
the application can be loaded. However, in terms of motivating and
explaining the application, you want to explain that support code
{\em last}.

The solution to this problem is to separate the Prolog loads from the
{\LaTeX} loads. This technique is used in the {\tt lib.pl} as
follows. Note in the following code segment, the use of
\verb+\input{libX}+ {\em after} the call to the Prolog
\verb+load_files+:

{\scriptsize \begin{verbatim}
%%%% Installation %<
:- load_files([lib0  % pre-load actions
              ,lib1  % predicates
              ,lib2  % start-up commands
              ],[silent(yes),if(changed)]).
%>
%\input{lib0}
%\input{lib1}
%\input{lib2}
\end{verbatim}}

One nice side-effect of this technique is that the {\PROD} source
code can be divided up into simple chunks. The files {\tt lib0.pl},
{\tt lib1.pl}, and {\tt lib2.pl} only contain {\PROD} body content
since the {\PROD} header, preamble and footer is only needed once in
the {\tt lib.pl} container file.
