% cd('/tim/src/pl/nbc'), [nbc].
%\documentclass[twocolumn,global]{svjour}
\documentclass[twocolumn,10pt]{book}
\usepackage{fancyhdr,lastpage,tex4log}

\oddsidemargin -1.5cm
\evensidemargin -1.5cm

\columnsep 0.3in
 \topmargin -2cm
 \textheight 26cm \textwidth 19cm

%\setlength{\headrulewidth}{0.0pt}
\setlength{\footrulewidth}{0.5pt}
\lfoot[\fancyplain{}{page~\protect\thepage~of~\protect\pageref{LastPage}}]%        % even pp, pageno on left
        {\fancyplain{}{}}%              % odd pages nothing on left
\rfoot[\fancyplain{}{}]%                % even pages nothing on right
       {\fancyplain{}{page~\protect\thepage~of~\protect\pageref{LastPage}}}%       % odd pp, pageno on right
\cfoot[\fancyplain{}{\itshape\leftmark}]% section on left
        {\fancyplain{}{\it }}
\rhead{}
\lhead{}\chead{}





%\journalname{Submitted to WVU Knowledge Engineering}
\date{Wp ref: tim/src/pl/nb/nb \today}
\newcommand{\ME}{NBC}
\begin{document}
\thispagestyle{plain}
\pagestyle{fancy}
\title{
\begin{minipage}{.2\linewidth}
\includegraphics[width=4cm]{nbc.eps}\end{minipage}\hfill
\begin{minipage}{.75\linewidth}
\ME: A Prolog-based Naive Bayes Classifier
\end{minipage}}
\author{Tim Menzies%\inst{1}
             , Santa Clause%\inst{2}
             }
%\institute{Lane Department of Computer Science,
       University of West Virginia,
       PO Box 6109, Morgantown,
       WV, 26506-6109, USA;\\
       \url{http://tim.menzies.us};
       \email{tim@menzies.us} \and
 Artic Software Systems} \maketitle
\thispagestyle{empty}
\maketitle
\pagestyle{fancy}
\setcounter{tocdepth}{4}
\tableofcontents \listoffigures
\thispagestyle{plain}
\pagestyle{fancy}
\chapter{To begin}
 \section*{Abstract}
{\em {\Tex4Log} is a simple set of macros written in {\LaTeX}
allowing for a simple documentation scheme for Prolog.}



\input{bayes}
\section{ Introductions
}
 The following package is a SWI-Prolog system~\cite{swiprolog}.
Prolog is a useful language for rapdily building
systems~\cite{der96,me89zb,bratko01,ok90,sterling94}.
This {\Tex4Log}
description  follows my standard(ish) pattern:
\bi
\item
Shell
\item Knowledge base
\item
An appendix with acknowledgements, references, and licensing details.
\ei
The shell divides up as follows:
\bi
\item Initializations:
\bi
\item
Operator definitions (must be first).
\item
Flags (these can usually go just before the start-up
actions but, for safety's sake, we place them at the
front).
\item
hooks (into the Prolog reader)
\item
hacks (shameful things we'd rather hide).
\ei
\item
The actual system code.
\item
Library code which, ideally,
should be good for more than just this application).
\item
Start-up code (must be loaded into Prolog last).
\ei

\section{ Initializations
}
\subsection{ Operators }\begin{Verbatim}
:- op(1001,xfx, the).
:- op(1001,xfx, a  ).
:- op(999 ,fx,  *  ).
:- op(700,xfx, :=).
:- op(1199 ,xfx, of ).
:- op(700,xfx, <- ).
:- op(1 ,fx, (?) ).
:- op(1 ,fx, (!) ).
\end{Verbatim}
\subsection{ Flags }\begin{Verbatim}
:- multifile     option/2, meta/9, oo/5, get/3,zap/3,commit/4.
:- dynamic       option/2, meta/9, oo/5, context/1.
:- discontiguous option/2, meta/9, oo/5, get/3,zap/3,commit/4.
\end{Verbatim}
\subsection{ Hooks }\begin{Verbatim}
term_expansion((W of X --> Y),Z) :- defmethod((W of X --> Y ),Z).
term_expansion((W of X),      Z) :- defmethod((W of X --> []),Z).
term_expansion((X the Rel),Z) :- dd(Rel,X,Z).
term_expansion((X a   Rel),Z) :- eg(Rel,X,Z).
term_expansion(A=B,       []) :- set(A=B).

goal_expansion(*(X,Y,Z),W) :-methodCall(*(X,Y,Z),W).
\end{Verbatim}
\subsection{ Hacks
}
 Shown here are some dark secrets of the Prolog
wizards. If you don't yet understand
the following code,
then you don't need to know it. Trust us,
we are knowledge engineers.  \begin{Verbatim}
goal_expansion(is(A,B,C,C),            is(A,B)).
goal_expansion(=(A,B,C,C),              =(A,B)).
goal_expansion(>=(A,B,C,C),            >=(A,B)).
goal_expansion(>(A,B,C,C),              >(A,B)).
goal_expansion(<(A,B,C,C),              <(A,B)).
goal_expansion(<=(A,B,C,C),            <=(A,B)).
\end{Verbatim}
\section{ Library code
}
\subsection{ Does a goal have only 1 matching clause? }\begin{Verbatim}
clause1(X,Y) :- singleton(X), clause(X,Y).

singleton(X) :-
    Sym='$singleton_',
    flag(Sym,_,0),
    \+ singleton1(Sym,X),
    flag(Sym,1,1).

singleton1(Sym,X) :- clause(X,_),flag(Sym,N,N+1),N > 1,!.
\end{Verbatim}
\subsection{ Does a goal have only 1 way to succeed? }\begin{Verbatim}
only(X) :-
    Sym='$only_',
    flag(Sym,_,0),
    \+ only1(Sym,X),
    flag(Sym,1,1).

only1(Sym,X) :- X, flag(Sym,N,N+1),N > 1,!.

solo(X) :- only(X), X.
\end{Verbatim}
\subsection{ Configuration Control }\begin{Verbatim}
[] := []    :- !.
[H0|T0] := [H|T] :- !, H0 := H, T0 := T.
X := Y      :-  option(X,Z),!, Y=Z.
X := _      :-  !, barph(missingOption(X)).
?X          :- atomic(X), X := 1.

set(X=Y) :- retractall(option(X,_)), assert(option(X,Y)).

commandLine :-
    current_prolog_flag(argv, Argv),
    append(_, [--|Args], Argv), !,
    concat_atom(Args, ' ', SingleArg),
    term_to_atom(Term, SingleArg),
    c2l(Term,List),
    forall(member(One,List), set(One)).
commandLine.
\end{Verbatim}
\subsection{ Demo support code
}

Catches the output from some predicate X
and saves it a file X.spy. The command:

{\scriptsize
\begin{verbatim}
\SRC{X.spy}{Caption}
\end{verbatim}}
\noindent
includes the generated file into the {\LaTeX} document.

The code `demos/1` deletes any old output and runs some goal twice: once
to trap it to a file and once to show the results on the screen.  \begin{Verbatim}
demos(G) :-
    sformat(Out,'~w.spy',G),
    (exists_file(Out) -> delete_file(Out) ; true),
    tell(Out),
    format('% output from '':- demos(~w).''\n\n',G),
        T1 is cputime,
    ignore(forall(G,true)),
    T2 is (cputime - T1),
    format('\n% runtime = ~w sec(s)\n',[T2]),
        told,
    format('% output from '':- demos(~w).''\n',G),
        ignore(forall(G,true)),
    format('\n% runtime = ~w sec(s)',[T2]).
\end{Verbatim}
\subsection{ Ordered counted key value pairs
}
\begin{Verbatim}
inc([],Key,[Key=1]).
inc([Key0=Value0|T],Key,Add) :-
        compare(Order,Key0,Key),
        inc(Order,Key0=Value0,T, Key,Add).

inc(<,H,     T,    X,[H       |Add]) :- inc(T,X,Add).
inc(=,Key=X, T,  Key,[Key=Y   |T])   :- Y is X + 1.
inc(>,H,     T,  Key,[Key=1,H |T]).

n([Key0=V|T],Key,X) :- compare(Ord,Key0,Key), n(Ord,V,T,Key,X).

n(=,V,_,_  ,V).
n(<,    _,T,Key,X) :- n(T,Key,X).
\end{Verbatim}
 Tagged pairs  \begin{Verbatim}
less1([H|T],H,T).
less1([_|T],Out,Rest) :- less1(T,Out,Rest).

key(L0,K,V0,V,[K=V|L]) :- less1(L0,K=V0,L).
\end{Verbatim}
\subsection{ Dump a Whole File to the Screen}\begin{Verbatim}
chars(F) :- see(F), get_byte(X), ignore(chars1(X)), seen.

chars1(-1) :- !.
chars1(X)  :- put(X), get_byte(Y), chars1(Y).
\end{Verbatim}
\subsection{ License }\begin{Verbatim}
hello :-  % \MARK{hello}
    [program,version,copyright,motto,copywho]:=[N,V,Y,M,C],
    format('~s version ~s\n Copyright (C) ~s by ~s\n',
           [N,V,Y,C]),
    format(' "~s"\n\n~s ~s ',[M,N,V]),
    chars('nowarranty.txt'). % see \tion{nowar}

warranty :-
    [program,copyright,copywho]:=[P, Y,C],
    format('~s by ~s\n Copyright (C) ~s\n\n',[P,C,Y]),
    chars('warranty.txt'),nl. % see \tion{war}

conditions :-
    chars('conditions.txt'),nl. % see \tion{cond}
\end{Verbatim}
\subsection{ Miscalleanous
}
 Pretty print a list of terms.  \begin{Verbatim}
portrays(L) :- portrays(L,_,_).

portrays([],_,_).
portrays([H|T],F0,A0) :-
    functor(H,F,A),
    (F0=F,A0=A
        -> portray_clause(H),
           portrays(T,F0,A0)
         ; nl,portray_clause(H),
           portrays(T,F,A)).
\end{Verbatim}
 Other stuff.  \begin{Verbatim}
times(N,G,Out) :-
    T1 is cputime, repeats(N,true),
    T2 is cputime, repeats(N,G),
    T3 is cputime, Out is (T3-T2-(T2-T1))/N.

repeats(N0,G) :-
    N is N0,
    forall(between(1,N,_),G).

c2l((X,Y),[X|Z]) :- !,c2l(Y,Z).
c2l(X,[X]).

l2c([W,X|Y],(W,Z)) :- l2c([X|Y],Z).
l2c([X],X).

mostC2l((X,Y),[X|Z]) :- !,mostC2l(Y,Z).
mostC2l(_,[]).

sneak(X) :- load_files(X,[silent(true),if(changed)]).

spit(N1,N2,X) :- (0 is N1 mod N2 -> spit(X) ; true).

spit(X) :- ?verbose,!,write(user,X),flush_output(user).
spit(_).

barph(X) :- format('%W> ~p\n',X),fail.

barphln(X) :-
    here(File,Line),
    format('%W> ~p@~p : ~p\n',[File,Line,X]),
        fail.

here(File,Line) :-
        source_location(Path,Line),
        file_base_name(Path,File).
\end{Verbatim}
\subsection{ Defining methods }\begin{Verbatim}
defmethod((W of X --> Y),[Z|Meta]) :-
    getContext(X,C),
    expandInContext(C,(W-->Y),Z0),
        tidy(Z0,Z),
    metaMethod(Z,C,Meta).

getContext(X,B) :- o([rel0_ = X,blank_ = B]),!.
getContext(X,_) :- barphln(X is unknown).

expandInContext(C,(W-->Y),Z) :-
    retractall(context(_)),
    assert(context(C)),
    expand_term((W --> Y),Z),
    retractall(context(_)).

metaMethod((X :- _),C,Ind) :- !, metaMethod1(X,C,Ind).
metaMethod(X,C,Ind) :- metaMethod1(X,C,Ind).

metaMethod1(X,C,[(:- index(Index)),(:- discontiguous F/A)]) :-
    functor(X,F,A),
    (A=2
        -> Index=.. [F,1,0],
           arg(1,X,C)
        ;  functor(Term,F,A),
       A1 is A - 1,
       arg(A1,X,C),
       Term =.. [F|L],
       append([1|Rest],[1,0],L),
       zeros(Rest),
       Index =.. [F|L]).

zeros([]).
zeros([0|T]) :- zeros(T).
\end{Verbatim}
 Remove stray `true`s.  \begin{Verbatim}
tidy(A,C) :-
    tidy1(A,B),
    (B = (Head :- true) -> C=Head ;  C=B).

tidy1(A,C) :- once(tidy2(A,C)).

tidy2(A,             A) :- var(A).
tidy2((A,B),    (A,TB)) :- var(A), tidy1(B,TB).
tidy2((A,B),    (TA,B)) :- var(B), tidy1(A,TA).
tidy2(((A,B),C),     R) :- tidy1((A,B,C), R).
tidy2((true,A),      R) :- tidy1(A,R).
tidy2((A,true),      R) :- tidy1(A,R).
tidy2((A;B),   (TA;TB)) :- tidy1(A,TA), tidy1(B,TB).
tidy2((A->B), (TA->TB)) :- tidy1(A,TA), tidy1(B,TB).
tidy2(not(A),  not(TA)) :- tidy1(A,TA).
tidy2((A :- B), R) :-
    tidy1(B,TB),(TB=true-> R=A; R=(A:-TB)).
tidy2((A,B),  R) :-
    tidy1(A,TA), tidy1(B,TB),(TB=true -> R=TA; R=(TA,TB)).
tidy2(A,A).
\end{Verbatim}
\subsection{ Wrapper
}
 Cool stuff  \begin{Verbatim}
methodCall(*(X,Y,Z),W) :-
    context(Y),
    wrapper(X,Y,Z,W).
\end{Verbatim}
\begin{Verbatim}
wrapper(X,A,B,Out) :-
    wrap(X,Before,[],After,[],Goal),
    append(Before,[Goal|After],Temp),
        adds2vars(Temp,A,B,Out).

adds2vars([X0],A,B,X) :- add2vars(X0,A,B,X).
adds2vars([X0,Y|Z],A,B,(X,Rest)) :-
    add2vars(X0,A,C,X),
    adds2vars([Y|Z],C,B,Rest).

add2vars(oo(X,Y,Z),A,B,oo(A,X,Y,Z,B)) :- !.
add2vars(X,A,A,X).
\end{Verbatim}
\begin{Verbatim}
wrap(X,B0,B,A0,A,Y) :-
    once(wrap0(X,Z)),
    wrap1(Z,B0,B,A0,A,Y).

wrap0(X,        leaf(X) ) :- var(X).
wrap0(X,        leaf(X) ) :- atomic(X).
wrap0([],    leaf(true) ).
wrap0([H|T],      [H|T] ).
wrap0(?X,           ?X  ).
wrap0(!X,           !X  ).
wrap0(X,        term(X) ).

wrap1(leaf(X),     B, B, A, A, X).
wrap1([H0|T0],     B0,B, A0,A, [H|T]) :-
    wrap(H0,   B0,B1,A0,A1,H),
    wrap(T0,   B1,B, A1,A, T).
wrap1(term(X),     B0,B, A0,A, Y) :-
    X =.. L0,
    wrap(L0,   B0,B,A0,A,L),
    Y =.. L.
wrap1(?X, [oo(X,Y,Y)|B],B,A,  A,Y).
wrap1(!X, B,B,[oo(X,_,Y)|A],A,Y).
\end{Verbatim}
\subsection{ Accessors
}
\subsubsection{ Usage }\begin{Verbatim}
o(Com) :- o(Com,X), X.
o(Com,X) :- o(Com,X,_).

o([],X,X).
o([H|T],X,Y)  :- o(H,X,Z), o(T,Z,Y).
o(F <- V,X,Y)   :- oo(X,F,_,V,Y).
o(F+V,X,Y)    :- oo(X,F,L,[V|L],Y).
o(F=V,X,X)    :- oo(X,F,V,V,X).

goal_expansion(o(Com),     (o(Com,_,X),X)).
goal_expansion(o(Com,X),   o(Com,X,_)).
goal_expansion(o([H],X,Y), o(H,X,Y)) :-
    nonvar(H).
goal_expansion(o([H1,H2|T],X,Y), (o(H1,X,Z),o([H2|T],Z,Y))) :-
    nonvar(H1).

goal_expansion(o(X,Y,Z),Body) :-
    clause1(o(X,Y,Z),Body).

goal_expansion(oo(X,F,V0,V,Y),_) :-
    \+ oo(X,F,V0,V,Y),
    barphln(unknown(F)).

% this works, but i dont trust it.
goal_expansion(oo(X,F,V0,V,Y),true) :-
    nonvar(F), % fail, % try un-commenting this to see if it is worth it.
    solo(oo(X,F,V0,V,Y)).
\end{Verbatim}
\subsubsection{ Field details
}
 Each field has annotations that indicate:
\be
\item If Prolog should index on a particular field.
\item A fields name and default falue.
\item Some rule that defines the
valid range of a field.
\ee  \begin{Verbatim}
detail(+X:R=D,1,      X,   R,   D).
detail(+X:R,  1,      X,   R,   _).
detail(+X=D,  1,      X,   any, D).
detail(+X,    1,      X,   any, _).
detail(X:R=D, 0,      X,   R,   D).
detail(X:R,   0,      X,   R,   _).
detail(X=D,   0,      X,   any, D).
detail(X,     0,      X,   any, _).
\end{Verbatim}
These details are stored in a `meta/9` fact which we
can manipulate as follows:   \begin{Verbatim}
oo(meta(A,B,C,D,E,F,G,H,I),X,Y,Z,Out) :- % \MARK{meta/5}
    meta_(X,Y,Z,meta(A,B,C,D,E,F,G,H,I),Out).

meta_(rel0_, X,Y,meta(X,B,C,D,E,F,G,H,I),meta(Y,B,C,D,E,F,G,H,I)).
meta_(arity_,X,Y,meta(A,X,C,D,E,F,G,H,I),meta(A,Y,C,D,E,F,G,H,I)).
meta_(rel_  ,X,Y,meta(A,B,X,D,E,F,G,H,I),meta(A,B,Y,D,E,F,G,H,I)).
meta_(index_,X,Y,meta(A,B,C,X,E,F,G,H,I),meta(A,B,C,Y,E,F,G,H,I)).
meta_(names_,X,Y,meta(A,B,C,D,X,F,G,H,I),meta(A,B,C,D,Y,F,G,H,I)).
meta_(rules_,X,Y,meta(A,B,C,D,E,X,G,H,I),meta(A,B,C,D,E,Y,G,H,I)).
meta_(inits_,X,Y,meta(A,B,C,D,E,F,X,H,I),meta(A,B,C,D,E,F,Y,H,I)).
meta_(blank_,X,Y,meta(A,B,C,D,E,F,G,X,I),meta(A,B,C,D,E,F,G,Y,I)).
meta_(vars_, X,Y,meta(A,B,C,D,E,F,G,H,X),meta(A,B,C,D,E,F,G,H,Y)).
\end{Verbatim}
\subsubsection{ Core engine
}
 Code for being able to  access and changed
named fields within a term.  \begin{Verbatim}
dd(Rel0,FieldsC,All) :-
    dd1s(new(Rel0,FieldsC),Meta),
    bagof(One, Meta^dd2(Meta,One), All).

dd1s(new(Rel0,Fields0),Out) :-
    atom_concat(Rel0,'_',Rel),
    c2l(Fields0,Fields1),
    reverse([+id_|Fields1], [_|Fields]),
    length(Fields,Arity),
    functor(Blank,Rel0,Arity),
    Blank =.. [_|Vars],
    dd1(Fields,
              meta(Rel0,Arity,Rel,[],[],[],[],Blank,Vars),
           Out).

dd1([],X,X).
dd1([H|T]) -->
    {detail(H,I,N,R,D)},
    o([index_ +I, names_ +N,rules_ +R,inits_ +D]),
    dd1(T).
\end{Verbatim}
\subsubsection{ dd2/2
}
 Reset the counter for this relation to zero.  \begin{Verbatim}
dd2(X,(:- print(reseting(Rel0)),nl,flag(Rel0,_,0))) :- o(rel0_ =Rel0,X).
\end{Verbatim}
 Create an index on the indexed fields.  \begin{Verbatim}
dd2(X,(:- index(Index))) :-
    o([rel0_ =Rel0,index_ =Index0],X),
    Index =.. [Rel0|Index0].
\end{Verbatim}
 Note that assertions in this relation may or may not
exist at a particular time.  \begin{Verbatim}
dd2(X,(:- dynamic R /A)) :- o([rel0_ =R,arity_ =A],X).
\end{Verbatim}
 Automatically generate arity five accessors, just like
the ones written manually at \Line{meta/5}.
\begin{Verbatim}
dd2(Meta,Out) :-
    o([names_ =Names,rel0_ =Rel0,rel_ =Rel,arity_ =Arity],Meta),
    nth0(Pos,Names,Name),
    length(Before,Pos),
    functor(Term0, Rel0,Arity), Term0 =.. [_|L0],
    functor(Term,  Rel0,Arity), Term  =.. [_|L1],
    append(Before,[Old|After],L0),
    append(Before,[New|After],L1),
    Out =.. [Rel,Name,Old,New,Term0,Term].
\end{Verbatim}
 Define a {\em bridge} predicate that calls the
arity five accessor that is relevant to a particular term.  \begin{Verbatim}
dd2(X,(oo(T0,Com,V0,V,T) :- Body)) :-
    o([rel0_ =Rel0,rel_ =Rel,arity_ =Arity],X),
    functor(T0,Rel0,Arity),
    Body =.. [Rel,Com,V0,V,T0,T].
\end{Verbatim}
 Finish up the meta-level fact.  \begin{Verbatim}
dd2(X,Y) :-
    o([rel0_ =Rel0,inits_ =Inits0],X),
    Inits =.. [Rel0|Inits0],
    o(inits_ <- Inits,X,Y).
\end{Verbatim}
 Write our terms with names fields very succinctly.  \begin{Verbatim}
dd2(X,(portray(Term) :- write(Rel/ Arity))) :-
    o([rel0_ =Rel,arity_ =Arity],X),
    functor(Term,Rel,Arity).
\end{Verbatim}
 Singlton accessors with arity 3 can be expanded to accessor 5   \begin{Verbatim}
dd2(X,(goal_expansion(H,Body)  :- clause1(H,Body))) :-
    o(rel_ =Rel,X),
    H =.. [Rel,_,_,_].
\end{Verbatim}
 Singleton accessors with arity 5 can be evaluated,
    then replaced with `true`.  \begin{Verbatim}
dd2(X,(goal_expansion(H,true) :- clause1(H,true))) :-
    o(rel_ =Rel,X),
    functor(H,Rel,5).
\end{Verbatim}
 Match a fact in the global db.  \begin{Verbatim}
dd2(X,(get(Rel,Vars,Term) :- Term)) :-
    indexVars(X,Rel,Vars,Term).
\end{Verbatim}
 Zap a fact in the global db \begin{Verbatim}
dd2(X,(zap(Rel,Vars,Term) :- retractall(Term))) :-
    indexVars(X,Rel,Vars,Term).
\end{Verbatim}
 Make a fresh fact in the global db.  \begin{Verbatim}
dd2(X,(commit(Rel,Vars,Y,Z) :- retractall(Y),assert(Z))) :-
    indexVars(X,Rel,Vars,Y),
    indexVars(X,Rel,Vars,Z).
\end{Verbatim}
 Create a term with the index variables pre-matched.  \begin{Verbatim}
indexVars(X,Rel,Ind,Term) :-
    o([rel0_ = Rel, arity_ = Arity, index_ = Nums],X),
    functor(Term,Rel,Arity),
    Term =.. [Rel|Args],
    indexVars1(Nums,Args,Ind).

indexVars1([],_,[]).
indexVars1([1|T],[X|Args],[X|Ind]) :- indexVars1(T,Args,Ind).
indexVars1([0|T],[_|Args],   Ind ) :- indexVars1(T,Args,Ind).
\end{Verbatim}
\subsubsection{ Demos }\begin{Verbatim}
egdd :- % \see{egdd.spy}
    expand_term(
       (+deptNo : num
        ,name   : [x,y,z]
        ,+age   : num = 1
        ,are the eg),
       X),
    portrays(X).
\end{Verbatim}
 \SRC{egdd.spy}{}

\subsubsection{ Generating an instance
}
 see \Line{d}  \begin{Verbatim}
eg(Rel,FieldsC,Out) :-
    mostC2l(FieldsC,FieldsL),
    flag(Rel,M,M+1),
    N is M + 1,
    spit(N,50,0),
    Datum =.. [Rel,N|FieldsL],
    (okDatum(Rel,Datum) -> Out = Datum ; Out=[]).

okDatum(X,B) :- o([rel0_ = X,blank_ = B]),!.
okDatum(_,_) :- barphln(badness).
\end{Verbatim}
\subsection{ Statistics
}
\subsubsection{ Gaussians }\begin{Verbatim}
stale=0, mean:num=0, n:num=0, sd:num=0,sum:num=0
,sumSquared:num=0, are the gaussian.

add(X) of gaussian -->
    * !stale=1,
    * !n is ?n + 1,
        * !sum is ?sum+X,
    * !sumSquared is ?sumSquared + X*X.

refresh of gaussian --> * ?stale=0,!.
refresh of gaussian -->
    * !mean is ?sum/ ?n,
    * ?sd is sqrt( ?sumSquared -( ?sum^2/ ?n)/( ?n-1)),
    * !stale=0.

sd(Sd) of gaussian -->
    refresh,
    * print(1),
    * ?sd = Sd.

egDefMethod :-
    SRC= (refresh of gaussian -->
            * !mean is ?sum/ ?n,
            * !sd is sqrt( ?sumSquared -( ?sum^2/ ?n)/( ?n-1)),
            * !stale=0),
    defmethod(SRC,Out),
    portrays(Out).
\end{Verbatim}
\subsubsection{ Gaussians }\begin{Verbatim}
bins=[], n: num=0, are the histogram.
add(X) of histogram -->  * inc(?bins,X,!bins).
        % better {prolog} stuff and hunt for ?x in ordinary
%stuff
\end{Verbatim}
\section{ Main }\begin{Verbatim}
%  go :-  relation := R, classSymbol := Goal,
 %        o([rel0_ = R,blank_ = B, names_ = Names]).

go :-   relation:=Rel, go(Rel).

go(Rel) :-
    setup(Rel,Names,Goal,Wme0),
    flag(Rel,Max,Max),
    go1(Max,Names,Goal,Rel,Wme0,_).

setup(Rel,Names,Goal,Wme) :-
    classSymbol := Goal,
    o([rel0_ =Rel,rules_ =Rules,names_= Names]),
    nth1(Pos,Names,Goal),
    nth1(Pos,Rules,Classes),
    maplist(setup1(Rel,Names),Classes,Wme).

setup1(Rel,Names,Class,Class=Term) :-
    o([rel0_ =Rel, blank_ = Term]),
    setup2(Names,Term).

setup2([],_).
setup2([Name|Names],Term) :-
        o([rel0_ = histogram,inits_ = H]),
    o(Name=H,Term,Term),
    setup2(Names,Term).

go1(0,_,_,_,W,W).
go1(Id,Names,Goal,Rel,W0,W) -->
    Id > 0,
    one(Rel,Id,Goal,One,Vars,Class),
    Id0 is Id - 1,
    less1(W0,Class=Counts0,W1),
    counts(Names,Vars,Counts0,Counts),
    go1(Id0,Names,Goal,Rel,[Class=Counts|W1],W).

one(Rel,Id,Goal,One,Vars,Class) :-
  o([rel0_ = Rel,blank_ = One, vars_ = Vars]),
  o([id_ =Id,Goal=Class], One,One),
  One.

counts([],_) --> [].
counts([Name|Names],[F|Fs]) --> count(Name,F), counts(Names,Fs).

count(Name,F,W0,W) :- true.

%count1(Name,Class) :-

\end{Verbatim}
\section{ The NB system
}
  If an application stands on some library,
    then there is some hope that the library
    may be useful elsewhere even if the application
is not. So, the first rule of Timm: a good application is an empty
application; i.e. is just a place where
supposedly reusable components work together for a while.

\subsection{ Loading }\begin{Verbatim}
namesData :- names,data.

names :- relation:=R, names(R).
data  :- relation:=R, data(R).

names(X) :- atom_concat(X,'.names',Y), [Y].
data(X)  :- atom_concat(X,'.data', Y), [Y].
\end{Verbatim}
\section{ Start up actions}\begin{Verbatim}
:- ['defaults.nbc' % see \tion{defaults}
   ,'config.nbc'   % see \tion{config}
    ].

:- commandLine.
:- ?verbose -> hello ; true.
:- namesData.
\end{Verbatim}
\section*{Acknowledgements}
\input{thanks}
\bibliographystyle{abbrv}{
\footnotesize \bibliography{refs}
\appendix
\input{license}
\end{document}
